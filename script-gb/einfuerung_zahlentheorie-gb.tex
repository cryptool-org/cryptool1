\documentclass[a4paper,11pt]{article}
% Da man keine Unterst"utzung der deutschen Sprache in diesem Fall braucht!!
% \usepackage[german]{babel}

\setlength{\fboxrule}{.5mm}
\setlength{\fboxsep}{1.75mm}
\setlength{\footnotesep}{6pt}
\usepackage{hyperref}
\usepackage{amsmath}
\usepackage{amsfonts}
\usepackage{amsthm}
\usepackage{makeidx}
\usepackage[pdftex]{graphicx}
% \usepackage{graphicx}
\makeindex
\hypersetup{colorlinks=true}
\usepackage[flushmargin]{footmisc}

\setlength{\textwidth}{460 pt}
\hoffset-0.7in
\newtheorem{definition}{Definition}
%\newcommand{\href}[2]{#2}
\newcommand{\hypertag}[2]{#1}
%\newcommand{\hyperlink}[2]{#2}
%\newcommand{\hypertarget}[2]{#2}
\newtheorem{theorem}{Theorem}
\newenvironment{Proof}[1]
  {\textbf{Proof #1} \\}
  {\hfill$\Box$ \\}

%\pdfinfo{ /Title (CrypTool script)
%/Creator (TeX)
%/Producer (pdfTeX 0.14a)
%/Author (Deutsch Bank AG)
%/CreationDate (D:20000920201000)
%}


\title{ CrypTool Script \\ Mathematics and Cryptography }


\author{
(c) The authors, 1998-2002 \\
Frankfurt am Main
}

\begin{document}
\pagestyle{plain}
\setlength{\fboxrule}{.5mm}
\setlength{\fboxsep}{1.75mm}
\setlength{\footnotesep}{6pt}
\addtolength{\footskip}{8pt}
%\setlength{\footskip}{4cm}
%\renewcommand{\footnoterule}{\parindent0cm\rule{13cm}{.1pt}\vspace{.2cm}}
%Formatierung der Fu"snoten

%space between text and footnote 
\renewcommand\footnoterule{%
  \vspace{2em}%   <-- one line space between text and footnoterule
%\kern-3\p@
  \hrule width .4\columnwidth
 \vspace{4pt}
%kern 2.6\p@
}

%\long\def\@makefntext#1{%
%    \parindent 1em%
 %   \noindent
%    \hbox to 1.8em{\hss\@makefnmark}#1}

\maketitle

\parskip 4pt
\vskip + 30 pt
{
In this CrypTool script you will find predominantly mathematically oriented
information for using encryption procedures. The main chapters have been written
by {\em various authors} and are therefore independent from one another. At the end of
most chapters you will find literature and URLs.

You will receive information about the principles of symmetrical and
asymmetrical encryption. A large section of the script is dedicated to the
fascinating topic of prime numbers. Using numerous examples, the elementary number
theory and modular arithmetic are introduced and applied in an exemplary 
manner in the RSA procedure.  After this, one receives insight into the mathematical ideas
behind modern cryptography. 

%You will also obtain an overview of the
%mathematical ideas behind modern cryptography.

A further chapter is devoted to digital signatures, which are an essential
component of e-business applications. The last chapter --- elliptic curves ---
fits in well with this. The mathematics of elliptic curves forms the basis for
rapid cryptographic algorithms for digital signatures; these algorithms are
extremely well suited for implementation on smartcards.

Whereas the program CrypTool teaches you how to use cryptography in practice, the
script provides those interested in the subject with a deeper understanding of
the mathematical algorithms used -- trying to do it didactically as good as possible.

The {\em authors} Bernhard Esslinger, Bartol Filipovic, Henrik Koy and Roger Oyono
would like to take this opportunity to thank their colleagues in the company and
at the universities of Frankfurt, Giessen and Karlsruhe. They particularly thank
Dr. Peer Wichmann from the Karlsruhe computer science research centre
(Forschungszentrum Informatik, FZI) for his down-to-earth support.

\enlargethispage{0.5cm}
As at CrypTool, the quality of the script is enhanced through your suggestions
and ideas for improvement. We look forward to your feedback.


\vskip +7 pt \noindent
You will find the current version of CrypTool under \newline
  \href{http://www.CrypTool.de}{\tt http://www.CrypTool.de},~~
  \href{http://www.cryptool.com}{\tt http://www.CrypTool.com}~~ or~~
  \href{http://www.cryptool.org}{\tt http://www.CrypTool.org}.
{
\vskip + 7 pt \noindent
The contact persons for this free tool are listed in the readme file for
CrypTool.}
}


\newpage
\tableofcontents
\addcontentsline{toc}{section}{Contents}
\newpage
\normalsize
\parindent0cm
\parskip4pt


\input{elementaryNumberTheory.inc}


\addcontentsline{toc}{section}{Index} 
\printindex
%
% ......................................................................................................................
%
% ~~~~~~~~~~~~~~~~~~~~~~~~~~~~~~~~~~~~~~~~~~~~~~~~~~~~~~~~~~~~~~~~~~~~~~~~~~~~~~~~~~~~~~~~~~~~~~~~~~~~~~~~~~~~~~~~~~~~~~
%

\end{document}


