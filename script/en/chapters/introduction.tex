% $Id$
% ............................................................................
%      V O R W O R T  und  E I N F � H R U N G (Zusammenspiel Skript-CT) 
% ~~~~~~~~~~~~~~~~~~~~~~~~~~~~~~~~~~~~~~~~~~~~~~~~~~~~~~~~~~~~~~~~~~~~~~~~~~~~


% --------------------------------------------------------------------------
\section*{Preface to the 6th Edition of the CrypTool Script}  \addcontentsline{toc}{section}{Preface to the 6th Edition of the CrypTool Script}

Starting in the year 2000 this script became a part of the 
CrypTool\index{CrypTool} package. It is designed to accompany the program 
CrypTool by explaining some mathematical topics in more detail, 
but still in a way which is easy to understand.

In order to also enable developers/authors to work together independently 
the topics have been split up and for each topic an extra chapter has been 
written which can be read on its own. The later editorial work in TeX added 
cross linkages between different sections and footnotes describing where you
can find the according functions within the CrypTool\index{CrypTool} 
program \hyperlink{appendix-menutree}{(see menu tree in appendix A).}
% in \ref{s:appendix-menutree}   AUFPASSEN, DASS nichts doppelt !!
% \hypertarget{appendix-menutree}{}\label{s:appendix-menutree}
Naturally there are much more interesting topics in mathematics and
cryptography which could be discussed in greater depth -- therefore this
is only one of many ways to do it.

\vspace{12pt}
Rapid spread of the Internet has also lead to intensified research in the
technologies involved, especially within the area of cryptography where a good
deal of new knowledge has arisen.

This edition of the script mainly updates the summaries of the following 
topical research areas:
\vspace{-7pt}
\begin{itemize}
  \item the search for the largest prime numbers (generalized Mersenne 
        and Fermat primes, ``M-40'') 
	\\ (chap. \ref{spezialzahlentypen}, \ref{zahlentyp_mersenne}), 
  \item progress in number theory (``Primes in P'') (chap. \ref{PrimesinP}), 
  \item the factorisation of big numbers (TWIRL, RSA-160)\index{TWIRL device} 
        (chap. \ref{TWIRLDevice}, \ref{RSA-160}),
  \item progress in cryptanalysis of the AES standard 
        (chap. \ref{NeueAES-Analyse}) and
  \item progress in brute-force attacks against symmetric algorithms 
        (chap. \ref{Brute-force-gegen-Symmetr}).
\end{itemize}

Chapter 5 (more about hash algorithms) and chapter 6 (now we explain in
greater detail how elliptic curves are defined over different number
fields) of this script have also been enhanced.

\vspace{12pt}
The first time the document was delivered with CrypTool\index{CrypTool} 
was in version 1.2.01. Since then it has been expanded and revised in every 
new version of CrypTool (1.2.02, 1.3.00, 1.3.02, 1.3.03 and now 1.3.04).

I'd be more than happy if this also continues in the further open-source
versions of CrypTool\index{CrypTool} (they will be delivered by the University
of Darmstadt).

I am deeply grateful to all the people helping with their impressive
commitment who have made this global project so successful. Especially I would
like to acknowledge the English language proof-reading of this script version
done by Richard Christenson and L. Montgomery.

I hope that many readers have fun with this script and that they get 
out of it more interest and greater understanding of this modern but 
also very ancient topic.
\\


Bernhard Esslinger

Hofheim, May 2003



% --------------------------------------------------------------------------
\newpage
\section*{Introduction -- How do the Script and the Program Play together?}  \addcontentsline{toc}{section}{Introduction -- How do the Script and the Program Play together?}


\textbf{This script}

This document is delivered together with the program CrypTool\index{CrypTool}. \par \vskip + 3pt

Because the articles in this script are largely self-contained, this text
can also be read independently of CrypTool\index{CrypTool}.

Chapters 4 and 6 require a deeper knowledge in mathematics, while the
other chapters should be understandable with a school leaving certificate.

The authors
% \hyperlink{appendix-authors}{(authors)}
% in \ref{s:appendix-authors}
% \hypertarget{appendix-authors}{}\label{s:appendix-authors}
have attempted to describe cryptography for a broad 
audience -- without being mathematically incorrect. We believe that this
didactical pretension is the best way to promote the awareness for IT
security and the readiness to use standardised modern cryptography.
\par \vskip + 15pt


\textbf{The program CrypTool\index{CrypTool}}

CrypTool\index{CrypTool} is a program with an extremely comprehensive online
help enabling you to use and analyse cryptographic procedures within a
unified graphical user interface.\par \vskip + 3pt

CrypTool\index{CrypTool} was developed during the end-user awareness program
at Deutsche Bank in order to increase employee awareness of IT security and provide them with
a deeper understanding of the term security.
A further aim has been to enable users to understand the cryptographic
procedures. In this way, using CrypTool
as a reliable reference implementation of the various encryption procedures
(because of using the industry-proven Secude Library\index{Secude}),
you can test the encryption implemented in other programs. \par \vskip + 3pt

CrypTool\index{CrypTool} is currently been used for 
training in companies and teaching at school and universities, and
moreover several universities are helping to further develop the project.
\par \vskip + 15pt


\textbf{Acknowledgment}

At this point I'd like to thank explicitly 3 people who especially 
contributed to CrypTool\index{CrypTool}. Without their talents and engagement 
CrypTool would not be what it is today:
\vspace{-7pt}
\begin{itemize}
   \item Mr.\ Henrik Koy
   \item Mr.\ J\"org Cornelius Schneider and
   \item Dr.\ Peer Wichmann.
\end{itemize}
Also I want to thank all the many people not mentioned here for their 
hard work (mostly carried out in their spare time).
\\

Bernhard Esslinger

Hofheim, May 2003

% Local Variables:
% TeX-master: "../script-en.tex"
% End:
