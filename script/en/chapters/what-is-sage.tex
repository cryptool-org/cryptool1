% ---------------------------------------------------------------------------
\newpage
\hypertarget{appendix-using-sage}{}
\section{Short Introduction into the Computer Algebra System SageMath}
\label{s:appendix-using-sage}
\index{SageMath}
\index{SageMath!code examples}

This book includes numerous code samples using SageMath. SageMath is an open
source computer algebra system (CAS) that supports teaching, study and
research in mathematics.  It combines many high-quality open
source packages\footnote{%
To get an impression of how big SageMath is: After downloading the source of SageMath 4.1,
it took around 5 hours on an average Linux PC to compile the whole system including
all libraries. The compiled version occupied 1.8 GB disk space.
}
and provides access to their functionalities via a
common interface, namely, a Python\footnote{%
There is also an easy interface to the C language, called Cython, which can be used
to substantially speed up functions in SageMath.\\
See \url{http://openwetware.org/wiki/Open_writing_projects/Sage_and_cython_a_brief_introduction}.
} based programming language.

\noindent SageMath can be used as a powerful desktop calculator, as a tool to help
(undergraduate) students study mathematics, or as a programming
environment for prototyping algorithms and research in algorithmic
aspects of mathematics.

\noindent You can get a quick impression of SageMath e.g. with the references in this footnote\footnote{%
- ``Invitation to Sage'' by David Joyner, last update 2009\\
\url{http://sage.math.washington.edu/home/wdj/teaching/calc1-sage/an-invitation-to-sage.pdf}\\
- ``The SDSU Sage Tutorial'',\\
  \url{http://www-rohan.sdsu.edu/~mosulliv/sagetutorial/}\\
  \url{http://www-rohan.sdsu.edu/~mosulliv/sagetutorial/sagecalc.html}\\
- ``SAGE For Newbies'' by Ted Kosan, 2007,\\
   \url{http://sage.math.washington.edu/home/tkosan/newbies_book/sage_for_newbies_v1.23.pdf}
}.
  
\noindent The official SageMath online documentation\footnote{%
  The corresponding official PDF documents can be downloaded at\\
  \url{http://www.sagemath.org/help.html}, \url{http://www.sagemath.org/doc} and \url{http://planet.sagemath.org}.
} is available at: \url{http://www.sagemath.org}.


%% \centering statt \begin{center} ... \end{center}, da dies zus�tzlich vertical space erzeugt.
\noindent In the meantime there are lots of PDF and HTML documents about using SageMath, so we name only a few of them as a good starting point\footnote{%
- ``Library'': {\centering \url{http://www.sagemath.org/library/index.html}},\\
- ``Documentation Project'': {\centering \url{http://wiki.sagemath.org/DocumentationProject}},\\
- ``Teaching'': {\centering \url{http://wiki.sagemath.org/Teaching_with_SAGE}}.
}.

\noindent With respect to studying cryptography, SageMath modules can be used
to complement a first course in cryptography\footnote{%
- Module sources in the directory \url{SAGE_ROOT/devel/sage-main/sage/crypto}.

\noindent\hangindent=6pt\makebox[6pt][l]{-}Overview, what crypto currently is in SageMath:\\
   \url{http://www.sagemath.org/doc/reference/sage/crypto/}

\noindent\hangindent=6pt\makebox[6pt][l]{-}Discussions
   about teaching related aspects of development crypto in SageMath:\\
   \url{http://groups.google.com/group/sage-devel/browse_thread/thread/c5572c4d8d42d081}

   % Leerzeile am Ende n�tig, sonst hat die zweite Zeile KEINEN h�ngenden Einzug?! (TODO)
   % Dadurch aber mehr Abstand von Seitennummer als sonst!
}.

\noindent Comprehensive introductions into cryptography are in this
footnote\footnote{%
- David Kohel's notes from 2008 are a ready cource\\
{\centering \url{http://www.sagemath.org/library/crypto.pdf} }
or the same eventually newer at\\
{\centering \url{http://sage.math.washington.edu/home/wdj/teaching/kohel-crypto.pdf} }.\\
- ``Introduction to Cryptography with Open-Source Software, a very good book
  from Alasdair McAndrew, CRC, 2011\\
}.


% ---------------------------------------------------------------------------
\section*{SageMath user interfaces}
SageMath is available free of charge and can be downloaded from the
following website:
\begin{center}
  \url{http://www.sagemath.org} \\
\end{center}
The default interface to SageMath is {\bf command line} based, as shown in
figure~\ref{fig:sage_cmd_interfaces}. However, there is a
graphical user interface to the software as well in the form of the
SageMath notebook (see figure~\ref{fig:sage_gui_interfaces}). We can even
use SageMath {\bf notebooks}\footnote{%
Further details about SageMath notebooks can be found at 
chapter~\ref{ec:Sage_Massierer} 
(``\nameref{ec:Implementing-for-Education}''
$\Rightarrow$ ``\nameref{ec:Sage_Massierer}'').
                      }
online at different servers, without having to install SageMath locally, e.g:
\begin{center}
\url{http://www.sagenb.org} or\\
\url{http://sage.mathematik.uni-siegen.de:8000}
\end{center}

SageMath runs under many Linux distributions, Mac OS X, and Windows.
For the Windows platform, a complete distribution of SageMath currently
only runs as a VMware image. 

\begin{figure}[!htpb]
\centering
\includegraphics[scale=0.6]{figures/sage-cmd}
\caption{SageMath command line interface}
\label{fig:sage_cmd_interfaces}
\end{figure}

\begin{figure}[!htpb]
\centering
\includegraphics[scale=0.4]{figures/sage-gui}
\caption[SageMath notebook interface]{SageMath notebook interface\footnotemark}
\label{fig:sage_gui_interfaces}
\end{figure}

\footnotetext{%
To start the SageMath gui locally: Enter notebook() at the SageMath prompt, and then your favorite
browser (Iceweasel, Firefox, IE, ...) is started e.g. with the URL \url{http://localhost:8000}.
}



% ---------------------------------------------------------------------------
\newpage
\section*{Getting help with using SageMath}

Upon loading SageMath from the command line, we are presented with
something similar to the following:
%
\begin{Verbatim}%
[fontsize=\footnotesize]
mnemonic:~$ sage
----------------------------------------------------------------------
| Sage Version 4.1, Release Date: 2009-07-09                         |
| Type notebook() for the GUI, and license() for information.        |
----------------------------------------------------------------------

sage: help
Type help() for interactive help, or help(object) for help about object.
sage:
sage:
sage: help()

Welcome to Python 2.6!  This is the online help utility.

If this is your first time using Python, you should definitely check out
the tutorial on the Internet at http://docs.python.org/tutorial/.

Enter the name of any module, keyword, or topic to get help on writing
Python programs and using Python modules.  To quit this help utility and
return to the interpreter, just type "quit".

To get a list of available modules, keywords, or topics, type "modules",
"keywords", or "topics".  Each module also comes with a one-line summary
of what it does; to list the modules whose summaries contain a given word
such as "spam", type "modules spam".
\end{Verbatim}
%
Plenty of help is provided in the form of the official SageMath
documentation that is distributed with every release of SageMath~(see
Figure~\ref{fig:sage_standard_doc}).
%
\begin{figure}[!htpb]
\centering
\includegraphics[scale=0.4]{figures/sage-online-doc}
\caption{The SageMath standard documentation}
\label{fig:sage_standard_doc}
\end{figure}
%
The official SageMath standard documentation includes the following documents:

\begin{itemize}
\item Tutorial --- This tutorial is designed to help SageMath beginners
  become familiar with SageMath. It covers many features that beginners
  should be familiar with, and takes one to three hours to go through.

\item Constructions --- This document is in the style of a SageMath
  ``cookbook''. It is a collection of answers to questions about
  constructing various objects in SageMath.

\item Developers' Guide --- This guide is for developers who want to
  contribute to the development of SageMath. Among other issues, it covers
  coding style and conventions, modifying the core SageMath libraries,
  modifying the SageMath standard documentation, and code review and
  distribution.

\item Reference Manual --- This manual provides complete documentation
  on the major features of SageMath. The description of a class is
  accompanied by numerous code samples. All code samples in the
  reference manual are tested before each SageMath release.

\item Installation Guide --- This guide explains how to install SageMath
  under various platforms.

\item A Tour of Sage --- This is a tour of SageMath that showcases various
  features of SageMath that are useful for beginners.

\item Numerical Sage --- This document introduces tools available
  under SageMath that are useful for numerical computation.

\item Three Lectures about Explicit Methods in Number Theory Using
  Sage --- This document is about using SageMath to perform computations
  in advanced number theory.
\end{itemize}

\noindent From within a SageMath session, we can obtain a list of commands matching
some pattern.  To do so, we type the first few characters and then
press the ``Tab'' key:
%
\begin{Verbatim}%
[fontsize=\footnotesize]
sage: Su[TAB]
Subsets                   Subwords                  SuzukiGroup
SubstitutionCryptosystem  SupersingularModule
\end{Verbatim}
%
If we know the exact name of a command, we can use the \texttt{help}
function to obtain further information on that command, or append the
question mark ``?'' to the command name.  For example,
the command \texttt{help(SubstitutionCryptosystem)} provides
documentation on the built-in class
\texttt{SubstitutionCryptosystem}. We can get documentation on
this class with the question mark as follows:
%
\begin{Verbatim}%
[fontsize=\footnotesize]
sage: SubstitutionCryptosystem?
Type:type
Base Class:<type 'type'>
String Form:<class 'sage.crypto.classical.SubstitutionCryptosystem'>
Namespace:Interactive
File:/home/mvngu/usr/bin/sage-3.4.1/local/lib/python2.5/site-packages/sage/crypto/classical.py
Docstring:

        Create a substitution cryptosystem.

        INPUT:

        - ``S`` - a string monoid over some alphabet

        OUTPUT:

        - A substitution cryptosystem over the alphabet ``S``.

        EXAMPLES::

            sage: M = AlphabeticStrings()
            sage: E = SubstitutionCryptosystem(M)
            sage: E
            Substitution cryptosystem on Free alphabetic string monoid
            on A-Z
            sage: K = M([ 25-i for i in range(26) ])
            sage: K
            ZYXWVUTSRQPONMLKJIHGFEDCBA
            sage: e = E(K)
            sage: m = M(``THECATINTHEHAT'')
            sage: e(m)
            GSVXZGRMGSVSZG

        TESTS::

            sage: M = AlphabeticStrings()
            sage: E = SubstitutionCryptosystem(M)
            sage: E == loads(dumps(E))
            True
\end{Verbatim}
%
\vspace{30pt}
For further assistance on specific problems, we can also
search the archive of the \texttt{sage-support} mailing list at
%
\begin{center}
  \url{http://groups.google.com/group/sage-support}
\end{center}



% ---------------------------------------------------------------------------
\newpage
\section*{Some examples using the built-in mathematical functions in SageMath}

Here are a few little examples\footnote{%
The examples are from the blog of Dr. Alasdair McAndrew, Victoria University,\\
\url{http://amca01.wordpress.com/2008/12/19/sage-an-open-source-mathematics-software-system}}
(all in console mode, for ease) to see what you can do with SageMath:

\begin{sagecode}
\begin{Verbatim}%
[fontsize=\footnotesize]
# * Calculus:
    sage: x=var('x')
    sage: p=diff(exp(x^2),x,10)*exp(-x^2)
    sage: p.simplify_exp()
     1024 x^10 + 23040 x^8 + 161280 x^6 + 403200 x^4 + 302400 x^2 + 30240

# * Linear Algebra:
    sage: M=matrix([[1,2,3],[4,5,6],[7,8,10]])
    sage: c=random_matrix(ZZ,3,1);c
     [ 7 ]
     [-2 ]
     [-2 ]
    sage: b=M*c
    sage: M^-1*b
     [ 7 ]
     [-2 ]
     [-2 ]

# * Number theory:
    sage: p=next_prime(randint(2^49,2^50));p
      1022095718672689
    sage: r=primitive_root(p);r
      7
    sage: pl=log(mod(10^15,p),r);pl
      1004868498084144
    sage: mod(r,p)^pl
      1000000000000000

# * Finite Fields (\url{http://en.wikipedia.org/wiki/Finite_field}):
    sage: F.<x>=GF(2)[]
    sage: G.<a>=GF(2^4,name='a',modulus=x^4+x+1)
    sage: a^2/(a^2+1)
      a^3 + a
    sage: a^100
      a^2 + a + 1
    sage: log(a^2,a^3+1)
      13
    sage: (a^3+1)^13
      a^2
\end{Verbatim}
\caption{Some small general samples in SageMath from different areas in mathematics}
\end{sagecode}






% ---------------------------------------------------------------------------
\newpage
\section*{Writing code samples with SageMath}

When you start using a CAS (computer algebra system) you normally type in the single commands
on the command line as in the above example\footnote{%
  The standard way for presenting SageMath code starts the lines with ``sage:'' and ``...''.
  \begin{Verbatim}%
  [fontsize=\footnotesize]
  sage: m = 11
  sage: for a in xrange(1, m):
  ....:     print [power_mod(a, i, m) for i in xrange(1, m)]
  ....:
  \end{Verbatim}

  \noindent This script usually uses the above convention for presenting SageMath code,
  if the code doesn't come from a SageMath script. When people copy and paste the SageMath code
  from this script, in order to enter it at the command line, 
  they should leave out ``sage:'' and ``...' from the script (nevertheless
  in most cases the command prompt can deal with these prefixes correctly).}.

But if you develop your own functions, modify them and call them, then it is much easier
to do the development in your own editor, save it to a script file and execute the functions
non-interactively on the command line manually.
Both ways to develop code were applied in
chapter \ref{CM_Sage_samples} (``\nameref{CM_Sage_samples}''), 
chapter \ref{PaP_Sage_samples} (``\nameref{PaP_Sage_samples}''), 
chapter \ref{primes:_Appendix_Sage-Samples} (``\nameref{primes:_Appendix_Sage-Samples}'') and in
chapter \ref{NumberTheory_Appendix_E} (``\nameref{NumberTheory_Appendix_E}'').

\noindent To program and test SageMath code using an editor there are two useful commands:
\verb!load()! and \verb!attach()!\footnote{%
See SageMath tutorial about Programming, chapter ``Loading and Attaching Sage files'',\\
\url{http://www.sagemath.org/doc/tutorial/programming.html\#loading-and-attaching-sage-files}.}.\\
Suppose you have a function definition like this:
\begin{Verbatim}%
   [fontsize=\footnotesize]
   def function(var1):
       r"""
       DocText.
       """
       ...
       return (L)
\end{Verbatim}
\noindent which has been saved to the file \texttt{primroots.sage}.

\noindent To load this function into SageMath (and do a syntax check at once),
use \verb!load()! as follows:

\texttt{sage: load primroots.sage}

\noindent and you can then proceed to use on the command line any variable or function
defined in that SageMath script\footnote{%
Notes:

\noindent\hangindent=6pt\makebox[6pt][l]{-}Don't use white spaces in your file name.

\noindent\hangindent=6pt\makebox[6pt][l]{-}Its recommended that your SageMath script
has the file extension
``.sage'', instead of ``.py''. With a SageMath script whose file name ends in
``.sage'', when you load it into SageMath then the default SageMath environment
is also loaded to make sure that it works as if you have defined your
function from the SageMath command line. This also applies if you run the
script from a bash shell using~~\texttt{\$ sage primroots.sage}.

\noindent\hangindent=6pt\makebox[6pt][l]{-}If you run your script as above,
then SageMath first parses your script,
writes it to another file called ``primroots.py'' (note the ``.py''
extension), adds all necessary variables to "primroots.py" as well as
writing any import statements to that file. That way, your SageMath script
is executed as if you had typed the definitions in your script to the
SageMath command line. An important difference isthat all output needs a
\verb!print! statement.

}.

Normally we also want to edit our own SageMath script and reload the
content of the changed script into SageMath again. In that case, you can
use the command \verb!attach()! (you also can apply \verb!attach()!
directly after loading the script, even before having changed the script;
and you can even omit \verb!load()!, as this is contained in \verb!attach()!):

\texttt{sage: attach primroots.sage}

Now edit the SageMath script in a text editor, but don't exit SageMath.
After you saved it within your text editor, the changed function
definition is reloaded into the running SageMath session after the
next typing of Enter (and a syntax check is done at once). This reloading is
done automatically for you, provided that all changes to your script
have been saved. You can think of the command \verb!attach()! as a way of
telling SageMath to watch for all changes to a file, and reloading the
file again once SageMath notices that there have been changes. With this
command, you don't have to copy and paste between your text editor and
the SageMath command line interface.

Here is a picture of SageMath code in the editor GVIM with activated code highlighting
(see figure~\ref{fig:sage-highlighted-code-in-editor}).
%
\begin{figure}[!htpb]
\centering
\includegraphics[scale=0.7]{figures/sage-highlighted-code-in-editor}
\caption{SageMath sample shown in an editor with code highlighting}
\label{fig:sage-highlighted-code-in-editor}
\end{figure}

\vspace{20pt}
If you prefer to see the output of an attached file as if you would have
typed in the commands on the commandline directly (not only what is shown
via \verb!print!) then you could use the command \verb!iload()!:
Each line is loaded one at a time. To load the next line, you have to
press the \verb!Enter! key. You have to repeatedly press the \verb!Enter! key
until all lines of the SageMath script are loaded into the SageMath session.

\texttt{sage: iload primroots.sage}


% \vspace{30pt}
\newpage
\noindent Some more hints:
\begin{itemize}
  \item To get the version of your SageMath environment: \texttt{version()}
  \item To move quickly to the SageMath code examples in this script,
    \begin{itemize}
      \item either look in the index at \verb#SageMath -> Code examples#,
      \item or have a look at the appendix ``\nameref{sc:List-of-Sage-Code-Examples}''.
    \end{itemize}
  \item The SageMath samples in this script are delivered with CrypTool.\\
        For further details see the end of the overview
        ``\nameref{sc:List-of-Sage-Code-Examples}''.
\end{itemize}

