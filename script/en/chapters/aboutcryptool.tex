% $Id$
% ............................................................................
%                 TEXT OF THE 2nd PAGE
% ~~~~~~~~~~~~~~~~~~~~~~~~~~~~~~~~~~~~~~~~~~~~~~~~~~~~~~~~~~~~~~~~~~~~~~~~~~~~

% --------------------------------------------------------------------------
\clearpage\phantomsection
\addcontentsline{toc}{chapter}{Overview}
\chapter*{Overview about the Content of the CrypTool Script}  

\parskip 4pt
%\vskip +12 pt
\noindent In this CrypTool\index{CrypTool} script you will find predominantly
mathematically oriented information on using cryptographic procedures. There is
also sample code in the CAS (computer algebra system) Sage for some of the
procedures.
The main chapters have been written by various {\bf authors}
(see appendix \ref{s:appendix-authors}) %\hyperlink{appendix-authors}{authors}
and are therefore independent from one another. At the end of most chapters
you will find references and web links.

The \hyperlink{Kapitel_1}{first chapter} explains the principles of symmetric
and asymmetric {\bf encryption} and describes shortly the current decryption
records of modern symmetric algorithms.

Because of didactic reasons the \hyperlink{Kapitel_PaperandPencil}
{second chapter} gives an exhaustive overview
about {\bf paper and pencil encryption methods}.

Big parts of this script are dedicated to the fascinating topic of 
{\bf prime numbers} (chap. \ref{Label_Kapitel_2}).
%\hyperlink{Kapitel_2}{{\bf prime numbers}}. 
Using numerous examples,
{\bf modular arithmetic} and 
{\bf elementary number theory} (chap. \ref{Chapter_ElementaryNT})
are introduced and applied in an exemplary manner for the {\bf RSA procedure}.

By reading chapter \ref{Chapter_ModernCryptography}
you'll gain an insight into the mathematical ideas and concepts behind 
{\bf modern cryptography}.

%A \hyperlink{Chapter_Hashes-and-Digital-Signatures}{further chapter}
Chapter \ref{Chapter_Hashes-and-Digital-Signatures} gives
an overview about the status of attacks against modern {\bf hash algorithms}
and is then shortly devoted to {\bf digital signatures}, 
which are an essential component of e-business applications.

Chapter \ref{Chapter_EllipticCurves} describes {\bf elliptic curves}:
they could be used as an alternative to RSA and in addition are extremely
well suited for implementation on smartcards.

The \hyperlink{Chapter_Crypto2020}{last chapter} {\bf Crypto2020}
discusses threats for existing cryptographic methods and introduces
alternative research approaches to achieve long-term security
of cryptographic schemes.

Whereas the \textit{e-learning program} CrypTool\index{CrypTool} motivates and
teaches you how to use cryptography in practice, the \textit{script} provides
those interested in the subject with a deeper understanding of the mathematical
algorithms used -- trying to do it in an instructive way.
If you are already a little bit familiar with this field of knowledge you can
gain a fast overview about the functions delivered by CrypTool looking at the
{\bf menu tree} (see appendix \ref{s:appendix-menutree}).

% Bernhard Esslinger, Matthias B\"uger, Bartol Filipovic, Henrik Koy, 
% Roger Oyono and J\"org Cornelius Schneider
The authors would like to take this opportunity to thank their colleagues 
in the company and at the universities of Frankfurt, Gie\ss en, 
Siegen, Karlsruhe and Darmstadt.

\enlargethispage{12pt}
As with the e-learning program CrypTool\index{CrypTool}, the quality of the 
script is enhanced by your suggestions and ideas for improvement. 
We look forward to your feedback.


% Local Variables:
% TeX-master: "../script-en.tex"
% End:
