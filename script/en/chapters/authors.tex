
% ++++++++++++++++++++++++++++++++++++++++++++++++++++++++++++++++++++++++++
\hypertarget{appendix-authors}{}
\subsection{Authors of the CrypTool Script\index{CrypTool}}
\label{s:appendix-authors}

This appendix lists the authors\index{Authors} of this document.\\
Please refer to the top of each individual chapter for their contribution.

\begin{description}

\item[Bernhard Esslinger,] \mbox{}\\
initiator of the CrypTool project, main author of this script, head IT
security at Deutsche Bank and lecturer on IT security at the University
of Siegen.
E-mail: besslinger@web.de, esslinger@fb5.uni-siegen.de.
 
\item[Matthias B\"uger,] \mbox{}\\ 
contributor to the chapter ``Elliptic Curves'', research analyst at
Deutsche Bank.

\item[Bartol Filipovic,] \mbox{}\\
original author of the CrypTool elliptic curve
implementation and the corresponding chapter in this script.

\item[Henrik Koy, ] \mbox{}\\
main developer and co-ordinator of CrypTool development
since version 1.3; script reviewer and \TeX\ guru; cryptographer 
and project leader IT at Deutsche Bank.

\item[Roger Oyono, ] \mbox{}\\
implementer of the CrypTool factorization dialog and
original author of chapter ``The Mathematical Ideas behind Modern
Cryptography''.

\item[J\"org Cornelius Schneider,] \mbox{}\\
design and support of CrypTool; crypto enthusiast and IT architect and
senior project leader IT at Deutsche Bank.

\item[Christine St\"otzel,] \mbox{}\\
Master of Business and Computer Science at the University of Siegen.

\end{description}





% ++++++++++++++++++++++++++++++++++++++++++++++++++++++++++++++++++++++++
\newpage
\hypertarget{appendix-movies}{}%neu
\subsection{Bibliography of Movies and Fictional Literature with Relation
to Cryptograpy, Books for Kids with Collections of Simple Ciphers}
\label{s:appendix-movies}
% {\bf movies and literature related to cryptography}(see appendix \ref{s:appendix-movies})
\index{Movies}
\index{Literature}


Cryptographic applications -- classical as well as modern ones -- have been
used in literature and movies. In some media they are only mentioned and
are a pure addmixture; in others they play a primary role and are explained
in detail; and sometimes the purpose of the story, which forms the framework,
is primarily to transport this knowledge and achieve better motivation.

Here is the beginning of an overview:

\begin{description}

\item[\textrm{[Poe1843]}] \index{Poe 1843}
    Edgar Allan Poe\index{Poe, Edgar Allan}, \\
    {\em The Gold Bug}, 1843. \\
    In this short story Poe tells as first-person narrator about his
    acquaintanceship with the curious Mr. Legrand. They detect a fabulous
    treasure via a gold bug and a vellum found at the coast of New England.\\
    The cipher consists of 203 cryptic symbols and it proves to be a
    general monoalphabetic substitution cipher (see
    chapter~\ref{monoalphabeticSubstitutionCiphers}).
    The story tells how they solve the riddle step by step using a combination
    of semantic and syntax analysis (frequency analysis of single letters in
    English texts).\\
    In this novel the code breaker Legrand says the famous statement:
    ``Yet it may be roundly asserted that human ingenuity cannot concoct a
    cipher which human ingenuity cannot resolve -- given the according
    dedication.''\\
    % Yet it may be roundly asserted that human ingenuity cannot concoct a
    % cipher which human ingenuity cannot resolve...

\item[\textrm{[Verne1885]}] \index{Verne 1885}
    Jules Verne\index{Verne, Jules}, \\
    {\em Mathias Sandorf}, 1885. \\
    This is one of the most famous novels of the French author Jules Verne
    (1828-1905), who was called ``Father of Science fiction''.\\
    In ``Mathias Sandorf'' he tells the story of the freedom fighter Earl
    Sandorf, who is betrayed to the police, but finally he can escape.\\
    The whistle-blowing worked, because his enemies captured and decrypted
    a secret message sent to him. For decryption they needed a special
    grille, which they stole from him. This turning grille was a quadratic
    piece of jig with 6x6 squares, of which 1/4 (nine) were holes
    (see the \hyperlink{turning-grille}{turning grille} in 
    chapter~\ref{introsamplesTranspositionCiphers}).\\


\item[\textrm{[Kipling1901]}] \index{Kipling 1901}
    Rudyard Kipling\index{Kipling, Rudyard}, \\
    {\em Kim}, 1901. \\
    Rob Slade's review%
    \footnote{See
          \href{http://catless.ncl.ac.uk/Risks/24.49.html\#subj12}
       {\texttt{http://catless.ncl.ac.uk/Risks/24.49.html\#subj12}}.
    }
    of this novel says:
    ``Kipling packed a great deal of information and concept into his stories,
    and in ``Kim'' we find The Great Game: espionage and spying.  Within the
    first twenty pages we have authentication by something you have, denial
    of service, impersonation, stealth, masquerade, role-based
    authorization (with ad hoc authentication by something you know),
    eavesdropping, and trust based on data integrity.  Later on we get
    contingency planning against theft and cryptography with key changes.''\\
    The book is out of copyright%
    \footnote{You can read it at:\\
          \href{http://whitewolf.newcastle.edu.au/words/authors/K/KiplingRudyard/prose/Kim/index.html}
       {\texttt{http://whitewolf.newcastle.edu.au/words/authors/K/KiplingRudyard/prose/Kim/index.html}},\\
          \href{http://kipling.thefreelibrary.com/Kim}
       {\texttt{http://kipling.thefreelibrary.com/Kim}} or\\
          \href{http://www.readprint.com/work-935/Rudyard-Kipling}
       {\texttt{http://www.readprint.com/work-935/Rudyard-Kipling}}.
    }.\\


\item[\textrm{[Doyle1905]}] \index{Doyle 1905}
    Arthur Conan Doyle\index{Doyle, Sir Arthur Conan}, \\
    {\em The Adventure of the Dancing Men}, 1905. \\
    In this Sherlock Holmes short story (first published in 1903 in the
    ``Strand Magazine'', and then in 1905 in the collection 
    ``The Return of Sherlock Holmes'' the first time in book-form)
    Sherlock Holmes has to solve a cipher which at first glance looks
    like a harmless kid's picture. \\
    But it proves to be the monoalphabetic substitution cipher (see
    chapter~\ref{monoalphabeticSubstitutionCiphers}) of the criminal Abe
    Slaney.
    Sherlock Holmes solves the riddle using frequency analysis.\\


\item[\textrm{[Sayer1932]}] \index{Sayer 1932}
    Dorothy L. Sayer, \\
    {\em Have his carcase}, Harper/Victor Gollancz Ltd., 1932. \\
    In this novel the writer Harriet Vane finds a dead body at the beach.
    The police believe the death is suicide.
    Harriet Vane and the elegant amateur sleuth Lord Peter Wimsey together
    clear of the disgusting murder in this second of Sayers's famous
    Harriet Vane mystery series. \\
    This requires to solve a cryptogram. Surprisingly the novel not only
    describes the Playfair cipher in detail, but also the cryptanalysis
    of this cipher
    (see \hyperlink{playfair}{Playfair} in 
    chapter~\ref{polygraphicSubstitutionCiphers}).\\


\item[\textrm{[Arthur196x]}] \index{Arthur 196x}
    Robert Arthur, \\
    {\em The Three Invesigators: The Secret Key (German version: Der
    geheime Schl\"ussel nach Alfred Hitchcock (volume 119)},
    Kosmos-Verlag (from 1960) \\
    The three detectives Justus, Peter and Bob have to decrypt covered and
    encrypted messages within this story to find out what is behind the toys
    of the Copperfield company.\\


\item[\textrm{[Seed1990]}] \index{Seed 1990}
    Directed by Paul Seed, \\
    {\em House of Cards}, 1990. \\
    In this movie Ruth tries to solve the secret, which made her daughter
    fall silent. Here two young people suffering from autism communicate
    via 5- and 6-digit primes (see 
    chapter~\ref{Label_Kapitel_2}).
    After more than 1 hour the movie contains the following undecrypted
    two series of primes:
    \begin{center}
    $21,383; \;\;176,081; \;\;18,199; \;\;113,933; \;\;150,377; \;\;304,523; \;\;113,933$ \\
    $193,877; \;\;737,683; \;\;117,881; \;\;193,877$
    \end{center}
    % where autistic children communicate via primes
    \vskip +20 pt

\item[\textrm{[Robinson1992]}] \index{Robinson 1992}
    Directed by Phil Alden Robinson, \\
    {\em Sneakers}, Universal Pictures Film, 1992. \\
    In this movie the ``sneakers'', computer experts under their boss Martin
    Bishop, try to get back the deciphering box SETEC from the ``bad guys''.
    SETEC, invented by a genius mathematician before he was killed, allows to
    decrypt all codes from any nation. \\
    The code is not described in any way.\\


\item[\textrm{[Becker1998]}] \index{Becker 1998}
    Directed by Harold Becker, \\
    {\em Mercury Rising}, Universal Pictures Film, 1998. \\
    The NSA developed a new cipher, which is pretended to be uncrackable by
    humans and computers. To test its reliability some programmers hide a
    message encrypted with this cipher in a puzzle magazine.\\
    Simon, a nine year old autistic boy, cracks the code. Instead of
    fixing the code, a government agent sends a killer. FBI agent Art
    Jeffries (Bruce Willis) protects the boy and sets a snare for the
    killers.\\    
    The code is not described in any way.\\


\item[\textrm{[Brown1998]}] \index{Brown 1998}
    Dan Brown, \\
    {\em Digital Fortress}, E-Book, 1998. \\
    Dan Brown's first novel was published in 1998 as e-book, but it was
    largely unsuccessful then.\\
    The National Security Agency (NSA) uses a huge computer, which enables it
    to decrypt all messages (needless to say only of criminals and terrorists)
    within minutes even if they use the most modern encryption methods.\\
    An apostate employee invents an unbreakable code and his computer program
    Diabolus forces the super computer to do self destructing operations.
    The plot, where also the beautiful computer expert Susan Fletcher has a
    role, is rather predictable.\\
    The idea, that the NSA or another secret service is able to decrypt any
    code, is currently popular on several authors: In ``Digital Fortress'' the
    super computer has 3 million processors -- nevertheless from todays sight
    this is by no means sufficient to hack modern ciphers.\\


\item[\textrm{[Elsner1999]}] \index{Elsner 1999}
    Dr.~C.~Elsner, \\
    {\em The Dialogue of the Sisters}, c't, Heise, 1999. \\
    In this short story, which is included in the CrypTool package
    \index{CrypTool} as PDF file, the sisters confidentially communicate
    using a variant of RSA (see chapter~\ref{rsabeweis} and the following).
    They are residents of a madhouse being under permanent surveillance.\\


\item[\textrm{[Stephenson1999]}] \index{Stephenson 1999}
    Neal Stephenson, \\
    {\em Cryptonomicon}, Harper, 1999. \\
    This very thick novel deals with cryptography both in WW2 and today.
    The two heroes from the 40ies are the excellent mathematician and
    cryptanalyst Lawrence Waterhouse, and the overeager and 
    morphium addicted US marine Bobby Shaftoe. 
    They both are members of the special allied unit 2702, which tries
    to hack the enemy's communication codes and at the same time to
    hide the own existance. \\
    This secretiveness also happens in the present plot, where the 
    grandchildren of the war heroes -- the dedicated programmer  
    Randy Waterhouse and the beautiful Amy Shaftoe -- team up. \\
    Cryptonomicon is notably heavy for non-technical readers in parts.
    Several pages are spent explaining in detail some of the concepts
    behind cryptography.
    Stephenson added a detailled description of the Solitaire cipher
    (see chapter~\ref{Further-PaP-methods}), a paper and pencil encryption
    algorithm\index{Paper- and pencil methods} developed by Bruce Schneier
    which is called ``Pontifex'' in the book. Another, modern algorithm
    called ``Arethusa'' is not explained in detail.\\


\item[\textrm{[Elsner2001]}] \index{Elsner 2001}
    Dr.~C.~Elsner, \\
    {\em The Chinese Labyrinth}, c't, Heise, 2001. \\
    In this short story, which is included in the CrypTool package
    \index{CrypTool} as PDF file, Marco Polo has to solve problems from
    number theory within a competition to become a major consultant of
    the Great Khan.\\


\item[\textrm{[Colfer2001]}] \index{Colfer 2001}
    Eoin Colfer, \\
    {\em Artemis Fowl}, Viking, 2001. \\
    In this book for young people the 12 year old Artemis, a genius thief,
    gets a copy of the top secret ``Book of the Elfs''. After he decrypted it
    with his computer, he finds out things, men never should have known. \\
    The used code is not described in detail or revealed.\\


\item[\textrm{[Howard2001]}] \index{Howard 2001}
    Directed Ron Howard, \\
    {\em A Beautiful Mind}, 2001. \\
    This is the film version of Sylvia Nasar's biography of the game theorist
    John Nash. 
    After the brilliant but asocial mathematician accepts secret work in 
    cryptography, his life takes a turn to the nightmarish.
    His irresistible urge to solve problems becomes a danger for himself and
    his family. Nash is -- within his belief -- a most important hacker
    working for the government.\\
    Details of his way analysing code are not described in any way.\\


\item[\textrm{[Apted2001]}] \index{Apted 2001}
    Directed by Michael Apted, \\
    {\em Enigma}, 2001. \\
    This is the film version of Robert Harris' ``historical fiction'' 
    {\em Enigma} (Hutchinson, London, 1995) about the World War II 
    codebreaking work at Bletchley Park in early 1943, when the actual
    inventer of the analysis Alan Turing (after Polish pre-work) already 
    was in the US.
    So the fictional mathematician Tom Jericho is the lead character
    in this spy-thriller.\\
    Details of his way analysing the code are not described.\\


\item[\textrm{[Isau2003]}] \index{Isau 2003}
    Ralf Isau, \\
    {\em The Museum of the stolen memories (original title: Das Museum
    der gestohlenen Erinnerungen)}, Thienemann-Verlag, 2003. \\
    In this exciting novel the last part of the oracle can only be
    solved with the joined help of the computer community.\\


\item[\textrm{[Brown2003]}] \index{Brown 2003}
    Dan Brown, \\
    {\em The Da Vinci Code}, Doubleday, 2003. \\
    The director of the Louvre is found murderered in his museum in
    front of a picture of Leonardo da Vinci. And the symbol researcher
    Robert Langdon is involved in a conspiracy.
    The plot mentions different classic codes (substitution like
    Caesar or Vigenere, as well as transposition and number codes).
    Also there are hints about Schneier and the sunflower.
    The second part of the book contains a lot of theologic considerations.\\
    This book has become one of the most widely read books of all time.\\


\item[\textrm{[McBain2004]}] \index{McBain 2004}
    Scott McBain, \\
    {\em Final Solution}, manuscript not pubished by Harper Collins, 2004
    (German version has been published in 2005). \\
    In a near future politicians, chiefs of military and secret services of
    many different countries take over all the power. With a giant computer
    network called ``Mother'' and complete surveillance they want to cement
    their power and commercialisation of life forever.
    Humans are only assessed according to their credit rating and globally
    acting companies elude of any democratic control.
    Within the thriller the obvious injustice, but also the realistic
    likelihood of this development are considered again and again.\\
    With the help of a cryptographer a code to destroy was built into the
    super computer ``Mother'': In a race several people try to start the
    deactivation (Lars Pedersen, Oswald Plevy, the female American president,
    the British prime minister and an unknown Finish named Pia, who wants to
    take revenge for the death of her brother). On the opposite side a killing
    group acts under the special guidance of the British foreign minister and
    the boss of the CIA.\\


\item[\textrm{[Burger2006]}] \index{Burger 2006}
    Wolfgang Burger, \\
    {\em Heidelberg Lies (original title: Heidelberger L\"ugen)}, Piper, 2006. \\
    This detective story playing in the Rhein-Neckar area in Germany has
    several independant strands and local stories, but mainly it is
    about Kriminalrat Gerlach from Heidelberg. On page 207 f. the
    cryptographic reference for one strand is shortly explained: The
    soldier H\"orrle had copied circuit diagrams of a new digital NATO
    decryption device and the murdered man had tried to sell his perceptions
    to China.\\



\item[\textrm{[Vidal2006]}] \index{Vidal 2006}
    Agustin Sanchez Vidal, \\
    {\em Kryptum}, Dtv, 2006. \\
    The first novel of the Spanish professor of art history has some
    similarities with Dan Brown's ``The Da Vinci Code'' from 2003, but allegedly
    Vidal started his writing of the novel already in 1996. Vidal's novel is
    a mixture between historic adventure and mystery thriller.
    It was a huge success in Spain and Germany.
    There is currently no English version available.\\
    In the year 1582 Raimundo Randa is waiting to be condemned to death -- he
    was all life long trying to solve a mystery.
    This mystery is about a pergament with cryptic characters, where a unique
    power is behind.
    Around 400 years later the American scientist Sara Toledano is fascinated
    by this power until she vanishes in Antigua.
    Her colleague, the crytographer David Calderon, and her daughter Rachel
    are searching for her and simultaneously they try to solve the code.
    But also secret organizations like the NSA chase after the secret
    of the ``last key''. They don't hesitate to kill for it.\\


\end{description}



\vskip +20 pt
Further samples of cryptology in fictional literature can be found on the
following German web page:
\begin{center}
    \href{http://www.staff.uni-mainz.de/pommeren/Kryptologie99/Klassisch/1\_Monoalph/Literat.html}
   {\texttt{http://www.staff.uni-mainz.de/pommeren/Kryptologie99/Klassisch/1\_Monoalph/}}\\
    \href{http://www.staff.uni-mainz.de/pommeren/Kryptologie99/Klassisch/1\_Monoalph/Literat.html}{\texttt{Literat.html}}
\end{center}
For some older authors (e.g. Jules Verne, Karl May, Arthur Conan Doyle,
Edgar Allen Poe) there are links to the original and relevant text pieces.\\




\vskip +60 pt
Kid books with collections of simpler cryptographic encryption methods, 
prepared in a didactical and exciting manner are in the following list
(please send us similar English kid books, because at the moment
our list contains only German kid books):

\begin{description}

\item[\textrm{[Mosesxxxx]}] \index{Moses xxxx}
    [no named author], \\
    {\em Top secret -- The Book for Detectives and Spies (original title:
    Streng geheim -- Das Buch f\"ur Detektive und Agenten)},
    Edition moses, [no year named]. \\
    This is a thin book for small kids with Inspector Fox and Dr. Chicken.\\


\item[\textrm{[Para1988]}] \index{Para 1988}
    Para, \\
    {\em Ciphers (original title: Geheimschriften)},
    Ravensburger Taschenbuch Verlag, 1988 (1st edition 1977). \\
    On 125 pages filled with a small font this mini format book explains
    many methods which young children can apply directly to encrypt or hide
    their messages. A little glossar and a short overview about the usage of
    encryption methods in history complete this little book.

    Right at page 6 it summarizes for beginners in an old fashion style
    ``The Important Things First'' about paper\&pencil encryption
    (compare chapter~\ref{Kapitel_PaperandPencil}):
    \begin{itemize}
      \item[-] ``It must be possible to encrypt your messages at any place and
               at any location with the easiest measures and a small effort
               in a short time.
      \item[-] Your cipher must be easy to remember and easy to read for your
               partners. But strangers should not be able to decrypt them.\\
               Remember: Fastness before finesse, security before carelessness.
      \item[-] Your message must always be as short and precise as a telegram.
               Shortness outranks grammer and spelling. Get rid of all needless
               like salutations or punctuation marks. Preferably use only
               small or only capital letters.''
    \end{itemize}
    \vskip +20 pt   % da "\\" hier nicht geht!


\item[\textrm{[M\"uller-Michaelis2002]}] \index{M\"uller-Michaelis 2002}
    Matthias M\"uller-Michaelis, \\
    {\em The manual for detectives. Everything you need to know about
     ciphers, codes, reading tracks and the biggest detectives of the world
     (original title: Das Handbuch f\"ur Detektive. Alles \"uber
     Geheimsprachen, Codes, Spurenlesen und die gro\ss en Detektive dieser
     Welt)}, S\"udwest, 2002. \\


\item[\textrm{[Kippenhahn2002]}] \index{Kippenhahn 2002}
    Rudolf Kippenhahn, \\
    {\em Top secret! -- How to encrypt messages and to hack codes (original
    title: Streng geheim! -- Wie man Botschaften verschl\"usselt und 
    Zahlencodes knackt)}, rororo, 2002. \\
    In this novel a grandpa, an expert for secret writings teaches his
    four grandchildren and their friends, how to encrypt messages which
    nobody should read. Because there is someone who hacks their secrets,
    the grandpa has to teach them more and more complicated methods. \\
    Within this story, which forms the framework, the most important classic
    encryption methods and its analysis are explained in a manner exciting
    and appropriate for children.\\


\item[\textrm{[Harder2003]}] \index{Harder 2003}
    Corinna Harder und Jens Schumacher, \\
    {\em Top secret. The big book for detectives (original title: 
     Streng geheim. Das gro\ss e Buch der Detektive)}, Moses, 2003. \\


\item[\textrm{[Flessner2004]}] \index{Flessner 2004}
    Bernd Flessner, \\
    {\em The Three Invesigators: Manual for Secret Messages (original
    title: Die 3 ???: Handbuch Geheimbotschaften)},
    Kosmos, 2004. \\
    On 127 pages you learn in an easy and exciting manner, structered by
    the method types, which secret languages (like the one of the Navajo
    indians or dialects) and which secret writings (real encryption or
    hiding via technical or liguistic steganography) existed and how simple
    methods can be decrypted.\\
    The author tells where in history the methods were used and in which
    novells authors used encryption methods [like in Edgar Allan Poe's
    ``The Gold Bug'', like with Jules Verne's hero Mathias Sandorf or like
    with Astrid Lindgren's master detective Blomquist who used the ROR language
    (similar inserting ciphers are the spoon or the B language)].\\
    This is a didactically excellent introduction for younger teens.\\


\end{description}

\mbox{}\\

If you know of futher literature and movies, where cryptography has a major
role or if you know of futher books, which address cryptography in a
didactical and for children adequate way, then we would be very glad if
you could send us the exact book titel and a short explanation
about the book's content. Thanks a lot.


\clearpage
\listoffigures
\addcontentsline{toc}{section}{\listfigurename}


\clearpage

\listoftables
\addcontentsline{toc}{section}{\listtablename}



