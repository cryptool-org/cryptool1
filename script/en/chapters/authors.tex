
% ++++++++++++++++++++++++++++++++++++++++++++++++++++++++++++++++++++++++++
\subsection{Authors of the CrypTool Script\index{CrypTool}}
\hypertarget{appendix-authors}{}\label{s:appendix-authors}

This appendix lists the authors\index{Authors} of this document.\\
Please refer to the top of each individual chapter for their contribution.

\begin{description}

\item[Bernhard Esslinger,] \mbox{}\\
initiator of the CrypTool project, main author of this script, head IT security
at Deutsche Bank and lecturer on IT security at the University of Siegen.
E-mail: besslinger@web.de.
 
\item[Matthias B\"uger,] \mbox{}\\ 
contribution to the chapter ``Elliptic Curves'', research analyst at Deutsche Bank.

\item[Bartol Filipovic,] \mbox{}\\
original author of the CrypTool elliptic curve
implementation and the corresponding chapter in this script.

\item[Henrik Koy, ] \mbox{}\\
main developer and co-ordinator of CrypTool development
since version 1.3; script reviewer and \TeX\ guru; cryptographer 
and project leader IT at Deutsche Bank.

\item[Roger Oyono, ] \mbox{}\\
implementer of the CrypTool factorization dialogue and
original author of chapter ``The Mathematical Ideas behind Modern
Cryptography''.

\item[J\"org Cornelius Schneider,] \mbox{}\\
design and support of CrypTool; crypto
enthusiast and senior project leader IT at Deutsche Bank.

\item[Christine St\"otzel,] \mbox{}\\
Master of Business and Computer Science at the University of Siegen.

\end{description}








% ++++++++++++++++++++++++++++++++++++++++++++++++++++++++++++++++++++++++
\newpage
\subsection{Bibliography of Movies and Literature with Relation to Cryptography}
\hypertarget{appendix-movies}{}
\label{s:appendix-movies}
% {\bf movies and literature related to cryptography}(see appendix \ref{s:appendix-movies})
\index{Movies}
\index{Literature}


Cryptographic applications -- classical as well as modern ones -- have been
used in literature and movies. In some media they are only mentioned and
are a pure addmixture; in others they play a primary role and are explained
in detail; and sometimes the the purpose of the story, which forms the framework,
is primarily to transport this knowledge and achieve better motivation.

Here is the beginning of an overview.

\begin{description}

\item[\textrm{[Doyle1905]}] \index{Doyle 1905}
    Arthur Conan Doyle \index{Doyle, Sir Arthur Conan}, \\
    {\em The Adventure of the Dancing Men}, 1905. \\
    In this Sherlock-Holmes short story (first published in 1903 in the
    ``Strand Magazine'', and then in 1905 in the collection 
    ``The Return of Sherlock Holmes'' the first time in book-form)
    Sherlock Holmes has to solve a cipher which at first glance looks
    like a harmless kid's picture. \\
    But it proves to be the
    monoalphabetic substitution cipher of the criminal Abe Slaney.
    Sherlock Holmes solves the riddle using frequent analysis.


\item[\textrm{[Sayer1932]}] \index{Sayer 1932}
    Dorothy L. Sayer, \\
    {\em Have his carcase}, Harper / Victor Gollancz Ltd., 1932. \\
    In this novel the writer Harriet Vane finds a dead body at the beach.
    The police believe the death is suicide.
    Harriet Vane and the elegant amateur sleuth Lord Peter Wimsey together
    clear of the repellent murder in this second of Sayers's famous
    Harriet Vane mystery series. \\
    This requires to solve a cryptogram. Surprisingly the novel not only
    describes the Playfair cipher in detail, but also the cryptanalysis
    of this cipher. 


\item[\textrm{[Seed1990]}] \index{Seed 1990}
    Directed by Paul Seed, \\
    {\em House of Cards}, 1990. \\
    In this movie Ruth tries to solve the secret, which made her daughter
    fall silent. Here two young people suffering from autism communicate
    via 5- and 6-digit primes.
    After more than 1 hour the movie contains the following undecrypted
    two series of primes:
    \begin{center}
    $21,383; \;\;176,081; \;\;18,199; \;\;113,933; \;\;150,377; \;\;304,523; \;\;113,933$ \\
    $193,877; \;\;737,683; \;\;117,881; \;\;193,877$
    \end{center}
    % where autistic children communicate via primes


\item[\textrm{[Robinson1992]}] \index{Robinson 1992}
    Directed by Phil Alden Robinson, \\
    {\em Sneakers}, Universal Pictures Film, 1992. \\
    In this movie the ``sneakers'', computer experts under their boss Martin
    Bishop, try to get back the deciphering box SETEC from the ``bad guys''.
    SETEC, invented by a genius mathematician before he was killed, allows to
    decrypt all codes from any nation. \\
    The code is not described in any way.


\item[\textrm{[Elsner1999]}] \index{Elsner 1999}
    Dr.~C.~Elsner, \\
    {\em The Dialogue of the Sisters}, c't, 1999. \\
    In this short story, which is included in the CrypTool package
    \index{CrypTool} as PDF file, the sisters confidentially communicate
    using a variant of RSA.


\item[\textrm{[Stephenson1999]}] \index{Stephenson 1999}
    Neal Stephenson, \\
    {\em Cryptonomicon}, Harper, 1999. \\
    This thick novel deals with cryptography both in WW2 and today.
    The two heros from the 40th are the excellent mathematician and
    cryptanalyst Lawrence Waterhouse, and the overeager and addicted
    US marine Bobby Shaftoe. 
    They both are members of the special allied unit 2702, which tries,
    to hack the enemy's communication codes and at the same time to
    hide the own existance. \\
    This secretiveness also happens in the present plot, where the 
    grandchidren of the war heros -- the dedicated programmer  
    Randy Waterhouse and the beautiful Amy Shaftoe -- team up. \\
    Cryptonomicon is notably heavy for non-technical readers in parts.
    Several pages are spent explaining in detail some of the concepts
    behind cryptography.
    Stephenson added a detailled description of the Solitaire cipher
    (called "Pontifex" in the book), a paper and pencil encryption
    algorithm\index{Paper- and pencil methods} developed by Bruce Schneier.
    The used modern algorithm is not revealed.


\item[\textrm{[Elsner2001]}] \index{Elsner 2001}
    Dr.~C.~Elsner, \\
    {\em The Chinese Labyrinth}, c't, 2001. \\
    In this short story, which is included in the CrypTool package
    \index{CrypTool} as PDF file, Marco Polo has to solve problems from
    number theory within a competition to become a major consultant of
    the Great Khan.


\item[\textrm{[Colfer2001]}] \index{Colfer 2001}
    Eoin Colfer, \\
    {\em Artemis Fowl}, Viking, 2001. \\
    In this book for young people the 12 year old Artemis, a genius thief, gets
    a copy of the top secret ``Book of the Elfs''. After he decrypted it with
    his computer, he finds out things, men never should have known. \\
    The used code is not described in detail or revealed.


\item[\textrm{[Howard2001]}] \index{Howard 2001}
    Ron Howard, \\
    {\em A Beautiful Mind}, 2001. \\
    This is the film version of Sylvia Nasar's biography of the game theorist
    John Nash. 
    After the brilliant but asocial mathematician accepts secret work in 
    cryptography, his life takes a turn to the nightmarish.
    His irresistible urge to solve problems become a danger for himself and
    his family. Nash is -- within his belief -- ein most important hacker working
    for the government.\\
    Details of his way analysing code are not described in any way.


\item[\textrm{[Apted2001]}] \index{Apted 2001}
    Directed by Michael Apted, \\
    {\em Enigma}, 2001. \\
    This is the film version of Robert Harris' ``historical fiction'' 
    {\em Enigma} (Hutchinson, London, 1995) about the World War II 
    codebreaking work at Bletchley Park in early 1943, when the actual
    inventer of the analysis Alan Turing (after Polish pre-work) already 
    was in the US.
    So the fictional mathematician Tom Jericho is the lead character
    in this spy-thriller.\\
    Details of his way analysing the code are not described.


\item[\textrm{[Kippenhahn2002]}] \index{Kippenhahn 2002}
    Rudolf Kippenhahn, \\
    {\em Top secret! -- How to encrypt messages and to hack codes (original
    title: Streng geheim! -- Wie man Botschaften verschl"usselt und 
    Zahlencodes knackt)}, rororo, 2002. \\
    In this novel a grandpa, an expert for secret writings teaches his
    four grandchildren and their friends, how to encrypt messages which
    nobody should read. Because there is someone who hacks their secrets,
    the grandpa has to teach them more and more complicated methods. \\
    Within this story, which forms the framework, the most important classic
    encryption methods and its analysis are explained in manner, exciting
    and appropriate for children.


\item[\textrm{[Isau2003]}] \index{Isau 2003}
    Ralf Isau, \\
    {\em The Museum of the stolen memories (original title: Das Museum
    der gestohlenen Erinnerungen)}, Thienemann-Verlag, 2003. \\
    In this novel the last part of the oracle can only be
    solved with the joined help of the computer community.


\item[\textrm{[Brown2003]}] \index{Brown 2003}
    Dan Brown, \\
    {\em The Da Vinci Code}, Doubleday, 2003. \\
    The director of the Louvre is found murderered in his museum in
    front of a picture of Leonardo da Vinci. And the symbol researcher
    Robert Langdon is involved in a conspiracy.
    The plot mentions different classic codes (substitution like
    Caesar or Vigenere, as well as transposition and number codes).
    Also there are hints about Schneier and the sunflower.
    The second part of the book contains a lot of theologic considerations.\\
    This book has become one of the most widely read books of all time.


\end{description}




