\newpage
\section{Encryption procedures}

\subsection{Encryption}

The purpose of encryption \index{Encryption} is to change data in such a way
that only an authorised recipient is able to reconstruct the plaintext. This has
the advantage that you can transmit encrypted data openly and nevertheless need
not fear a perpetrator reading the data without authorisation. Authorised
recipients possess a piece of secret information --- called the key --- which
allows them to decrypt the data while it remains hidden from everyone else.\par \vskip + 3pt

One encryption procedure has been proved to be secure --- the {\em One Time
  Pad}.
\index{One �Time �Pad} However, this procedure has several practical
disadvantages (the key used must be selected randomly and must be just as long
as the message to be protected), which means that it is hardly used except in
closed environments such as for the hot wire between Moscow and Washington.\par \vskip + 3pt

For all other procedures there is a (theoretical) possibility of breaking them.
If the procedures are good, however, the time taken to break them is so long
that it is practically impossible to do and these procedures can therefore be
considered (practically) secure.\par \vskip + 3pt

We basically distinguish between symmetric and asymmetric encryption procedures.

\subsubsection{Symmetric encryption}

For {\em symmetric} encryption \index{Encryption!symmetric} the sender and
recipient must possess a common (secret) key which they have exchanged before
actually starting to communicate. The sender uses this key to encrypt the
message and the recipient uses it to decrypt it.\par \vskip + 3pt

The advantages of symmetric algorithms are the high speed with which data can be
encrypted and decrypted. One disadvantage is the need for key management. In
order to communicate with one another confidentially, sender and recipient must
have exchanged a key using a secure channel before actually starting to
communicate. Spontaneous communication between individuals who have never met
therefore seems virtually impossible. If everyone wants to communicate with
everyone else spontaneously at any time in a network of $ n $ subscribers, each
subscriber must have previously exchanged a key with each of the other $n-� 1$
subscribers. A total of $n(n - 1)/2$ keys must therefore be exchanged.\par \vskip + 3pt

The most well-known symmetric encryption procedure is the \index{DES} DES-algorithm. The DES-algorithm has been developed by IBM in collaboration with the
National Security Agency \index{NSA} (NSA), and was published as a standard in
1975. Despite the fact that the procedure is relatively old, no effective attack
on it has yet been detected. The most effective way of attacking consists of
testing all possible keys until the right one is found ({\em brute--force--attack}).
\index{Attack!brute-force} Due to the relatively short key length of
effectively 56 bits (64 bits, which however include 8 parity bits), numerous
messages encrypted using DES have in the past been broken. Therefore, the
procedure can now only be considered to be conditionally secure. Symmetric
alternatives to the DES procedure include the IDEA \index{IDEA} or Triple DES
algorithms.\par \vskip + 3pt

Up-to-the-minute procedures are the symmetric AES procedures. The associated
Rijndael procedure was declared winner the AES award on 2 October 2000 and thus
succeeds the DES procedure.

\subsubsection{Asymmetric encryption}

In the case of {\em asymmetric} encryption \index{Encryption!asymmetric} each
subscriber has a personal pair of keys consisting of a {\em secret}
\index{Key!secret} key and a {\em public} key\index{Key!public}. The public
key, as its name implies, is made public, e.g. in a key directory on the
Internet.\par \vskip + 3pt

If Alice wants to communicate with Bob, then she finds Bob's public key in the
directory and uses it to encrypt her message to him. She then sends this
ciphertext to Bob, who is then able to decrypt it again using his secret key. As
only Bob knows his secret key, only he can decrypt messages addressed to him.
Even Alice who sends the message cannot restore plaintext from the (encrypted)
message she has sent. Of course, you must first ensure that the public key
cannot be used to derive the private key.\par \vskip + 3pt

Such a procedure can be demonstrated using a series of thief-proof letter boxes.
If I have composed a message, I then look for the letter box of the recipient
and post the letter through it. After that, I can no longer read or change the
message myself, because only the legitimate recipient possesses the key for the
letter box.\par \vskip + 3pt

The advantage of asymmetric procedures is the easy \index{Key
management} key management. Let's look again at a network with $n$
subscribers. In order to ensure that each subscriber can establish
an encrypted connection to each other subscriber, each subscriber
must possess a pair of keys. We therefore need $2n$ keys or $n$
pairs of keys. Furthermore, no secure channel is needed before
messages are transmitted, because all the information required in
order to communicate confidentially can be transmitted openly. In
this case, you simply have to pay attention to the accuracy
(integrity and authenticity) \index{Authenticity} of the public
key. Disadvantage: Pure asymmetric procedures take a lot longer to
perform than symmetric ones.\par \vskip + 3pt

The most well-known asymmetric procedure is the \index{RSA} RSA algorithm,
named after its developers Ronald \index{Rivest Ronald} Rivest, Adi
\index{Shamir Adi} Shamir and Leonard \index{Adleman Leonard} Adleman. The RSA algorithm
was published in 1978. The concept of asymmetric encryption was first
introduced by Whitfield Diffie \index{Diffie Whitfield}  and Martin
\index{Hellman Martin} Hellman in 1976. Today, the ElGamal \index{ElGamal}
procedures also play a decisive role, particularly the \index{Schnorr} Schnorr
variant in the \index{DSA} DSA (Digital \index{Signatures!digital}Signature
Algorithm).

\newpage
\subsubsection{Hybrid procedures}
\index{Hybrid procedures}
In order to benefit from the advantages of symmetric and asymmetric techniques
together, hybrid procedures are usually used (for encryption) in practice.\par \vskip + 3pt

In this case the data is encrypted using symmetric procedures: the key is a
session key\index{Session key} generated by the sender randomly that is only used for this message.
This session key is then encrypted using the asymmetric procedure and
transmitted to the recipient together with the message. Recipients can determine
the session key using their secret keys and then use the session key to encrypt
the message. In this way, we can benefit from the easy key management \index{Key
management} of asymmetric procedures and encrypt large quantities of data
quickly and efficiently using symmetric procedures.

\subsubsection{Further details}

Beside information you can find in many books and on a lot of websites the online help
of CrypTool also offers very many details about the symmetric and asymmetric encryption methods.

\begin{thebibliography}{99999}
\addcontentsline{toc}{subsection}{Literatur}
\bibitem[Schmeh2001]{Schmeh2001}  \index{Schmeh 2001}
        Klaus Schmeh, \\
        Kryptografie und Public-Key Infrastrukturen im Internet, dpunkt.verlag, 2. akt. und erw. Auflage 2001. \\
        A considerable, up-to-date, good reading book, which considers practical problems, like standardisation or
        real existing software. Currently published only in German language.
\end{thebibliography}

