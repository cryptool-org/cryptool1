% $Id%
\pagebreak
\enlargethispage{1cm}
\section{CrypTool-Men"us}
Dieser Anhang enth"alt den kompletten Men"u-Baum von CrypTool. Bitte beachten
Sie, dass nicht immer alle Men"ueintr"age gleichzeitig verf"ugbar sind. Welche
gerade angezeigt werden bestimmt das ge"offnete Dokumentenfenster. So ist z.~B.
die Brute-Force Analyse f"ur DES nur verf"ugbar, wenn das aktive Fenster in
Hexadezimal-Darstellung ge"offnet ist, w"ahrend die Funktion "`Zufallsdaten
erzeugen"' immer verf"ugbar ist. In den Abbildungen \ref{menu-detail-1} und
\ref{menu-detail-2} ist dieser Zusammenhang in eckigen Klammern dargestellt,
z.~B. bedeutet "`Suchen\dots[ATH]"', dass die Funktion "`Suchen\dots"' nur f"ur
ASC-, Text- und Hexadezimal-Fenster verf"ugbar ist.
\begin{center}
\begin{tabular}{rl}
\bf Codebuchstabe & \bf Dokumententyp \\
M & (kein Dokument ge"offnet)\\
A & ASC\\
T & Text\\
H & Hexadezimal\\
P & Plot\\
\end{tabular}
\end{center}

\nobreak

\begin{figure}[!hb]
\begin{center}
\includegraphics[scale=1.1, clip, viewport=10 360 400 600]{figures/cryptool-menu-detail-de}
\caption{Detaillierte Ansicht des CrypTool Men"u-Baums -- Teil 1}
\label{menu-detail-1}
\end{center}
\end{figure}

\begin{figure}[b]
\begin{center}
\includegraphics[scale=1.1, clip, viewport=400 190 784 598]{figures/cryptool-menu-detail-de}
\caption{Detaillierte Ansicht des CrypTool Men"u-Baums -- Teil 2}
\label{menu-detail-2}
\end{center}
\end{figure}

\begin{figure}[hb]
\begin{center}
\includegraphics[scale=0.75, angle=270, viewport=14 107 779 590]{figures/cryptool-menu-de}
\vspace{-18pt}
\caption{"Ubersicht des CrypTool Men"u-Baum} 
\label{menuoverview}
\end{center}
\end{figure}
