% $Id%

% ++++++++++++++++++++++++++++++++++++++++++++++++++++++++++++++++++++++++++
\newpage
\enlargethispage{1cm}
\hypertarget{appendix-menutree}{}
\section{CrypTool-Men�baum}
\label{s:appendix-menutree}

   % Eyecatcher_Neue-CrypTool-Version
Dieser Anhang enth�lt auf der folgenden Seite den kompletten Men�baum der
CrypTool\index{CrypTool}-Version 1.4.31\footnote{%
  Parallel zu CrypTool 1.x (CT1)\index{CrypTool 1.x} werden im CrypTool-Projekt
  momentan auch die Zukunftsversionen CrypTool 2 (CT2)\index{CrypTool 2.0} und
  JCrypTool (JCT)\index{JCrypTool 1.0} entwickelt:\\
  - Webseite CT2: \url{http://www.cryptool2.vs.uni-due.de} \\
  - Webseite JCT: \url{http://jcryptool.sourceforge.net} \\
  Diese Zukunftsversionen sind zur Zeit (Dezember 2009) noch Betaversionen;
  sie sind aber schon stabil genug, um von Endbenutzern genutzt werden
  zu k�nnen.
}.

\noindent Das Haupt-Men� von CT1 enth�lt die Service-Funktionen in den Men�s
\begin{itemize}
   \item Datei
   \item Bearbeiten
   \item Ansicht
   \item Optionen
   \item Fenster
   \item Hilfe
\end{itemize}
und die eigentlichen Krypto-Funktionen in den Men�s
\begin{itemize}
   \item Ver-/Entschl�sseln
   \item Digitale Signaturen/PKI
   \item Einzelverfahren
   \item Analyse.
\end{itemize}

Unter \verb#Einzelverfahren# finden sich auch die Visualisierungen von Einzelalgorithmen
und von Protokollen. Manche Verfahren sind sowohl als schnelle Durchf�hrung
(meist unter dem Men� \verb#Ver-/Entschl�sseln#) als auch als
Schritt-f�r-Schritt-Visualisierung implementiert.

Welche Men�eintr�ge gerade aktiv (also nicht ausgegraut) sind, wird durch
den Typ des aktiven Dokumentfensters bestimmt:
So ist z.~B. die Brute-Force-Analyse\index{Angriff!Brute-Force} f�r DES 
nur verf�gbar, wenn das aktive Fenster in Hexadezi"-mal-Darstellung 
ge�ffnet ist, w�hrend der Men�eintrag "`Zufallsdaten erzeugen\dots"'
immer verf�gbar ist (auch wenn kein Dokument ge�ffnet ist). 

%Folgende vier Dokumenttypen gibt es in CrypTool:
%\begin{center}
%\begin{tabular}{rl}
%\bf Codebuchstabe & \bf Dokumententyp \\
%T & Textdatei-Ansicht\\
%H & Hexadezimal-Ansicht\\
%P & Diagramm/Plot-Ansicht (Histogramm, Autokorrelation)\\
%O & OpenGL Graphics-Ansicht\\
%\end{tabular}
%\end{center}


\clearpage
\begin{figure}[hb]
\begin{center}
\vspace{-30pt}
%\frame{
%\includegraphics[scale=0.25, angle=270, viewport=200 30 2680 1420]
\includegraphics[scale=0.35, angle=270]
                {figures/CT1-menutree-de}
%viewport=rand-links? rand-unten breite hoehe [bezogen auf querformat]
%}
\caption{Komplette �bersicht �ber den Men�-Baum von CT1 (CrypTool 1.4.31)} 
\label{menuoverview}
\end{center}
\end{figure}
\clearpage


\noindent Text zu CT2 xxxxxxxxxx DRAFT

\clearpage
\begin{figure}[hb]
\begin{center}
\vspace{-30pt}
\includegraphics[scale=0.8, angle=270]
                {figures/CT2-templatetree-de}
\caption{�bersicht �ber den Template-Baum von CT2 (NB4882.1, Juli 2012), Teil 1} 
\label{menuoverview}
\end{center}
\end{figure}
\clearpage

