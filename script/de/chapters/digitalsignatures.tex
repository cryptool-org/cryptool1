% $Id$
% !Mode:: "TeX:DE"    % Setting document mode and submode for WinEdt
% ..............................................................................
%                  D I G I T A L E  S I G N A T U R E N
% ~~~~~~~~~~~~~~~~~~~~~~~~~~~~~~~~~~~~~~~~~~~~~~~~~~~~~~~~~~~~~~~~~~~~~~~~~~~~~~
% be_2016-07-13: Housekeeping: " \\\" --> "\\" done (d+E).

\begin{refsegment}

% --------------------------------------------------------------------------
\hypertarget{Chapter_Hashes-and-Digital-Signatures}{}
\chapter{Hashfunktionen und Digitale Signaturen}
\label{Chapter_Hashes-and-Digital-Signatures}
\index{Signatur!digital}
(\hyperlink{author_Joerg-Cornelius-Schneider}{Jörg-Cornelius Schneider} /
 \hyperlink{author_Bernhard-Esslinger}{Bernhard Esslinger} /
 \hyperlink{author_Henrik-Koy}{Henrik Koy},
 Juni 2002;
 Updates: Feb. 2003, Juni 2005, Juli 2009, Nov. 2012)
% --------------------------------------------------------------------------
\begin{ctsquote}
Wir können alles aus dieser Welt machen, nur nicht eine Welt, in der die
Menschen in einigen zigtausend Jahren überlegen könnten: 'So, es ist nun
genug. So soll es von nun an für immer bleiben. Verändern wir nichts,
erfinden wir nichts, weil es besser nicht sein kann, und wenn doch, dann
wollen wir es nicht.'
\caption[Stanislaw Lem]{Stanislaw Lem\footnotemark}\index{Lem, Stanislaw}
\end{ctsquote}
\addtocounter{footnote}{0}\footnotetext{Antwort von Stanislaw Lem auf
die Kritik an seinem philosophischen Hauptwerk ~\glqq Summa Technologiae\grqq,
1964, in der er die evolutionäre Möglichkeit einer Entstehung der
künstlichen Intelligenz ausführte.}


Ziel der digitalen Signatur ist es, folgende zwei Punkte zu gewährleisten:
\begin{itemize}
 \item Benutzerauthentizität:\index{Authentizität!Benutzer-}\\
      Es kann überprüft werden, ob eine Nachricht tatsächlich
      von einer bestimmten Person stammt.
 \item Nachrichtenintegrität:\index{Nachrichtenintegrität}\\
      Es kann überprüft werden, ob die Nachricht (unterwegs)
      verändert wurde.
\end{itemize}


Zum Einsatz kommt wieder eine asymmetrische Technik (siehe
Verschlüsselungsverfahren).
Ein Teilnehmer, der eine digitale Signatur für ein Dokument erzeugen will,
muss ein Schlüssel"-paar besitzen. Er benutzt seinen geheimen Schlüssel,
um Signaturen zu erzeugen, und der Empfänger benutzt den öffentlichen
Schlüssel des Absenders, um die Richtigkeit der Signatur zu überprüfen.
Es darf wiederum nicht möglich sein, aus dem öffentlichen den geheimen
Schlüssel abzuleiten.

Im Detail sieht ein \index{Signaturverfahren}
{\em Signaturverfahren}\footnote{%
Mit CT1\index{CT1} können Sie ebenfalls digitale Signaturen
erzeugen und prüfen:\\
entweder in den Untermenüs des Hauptmenüpunktes
\textbf{Digitale Signaturen / PKI} oder\\
per
\textbf{Einzelverfahren \textbackslash{} RSA-Kryptosystem \textbackslash{}
Signaturdemo (Signaturerzeugung)}.\\
Ebenso können mit JCT\index{JCT} (in der Standard- und der
Algorithmen-Perspektive) unterschiedliche Arten elektronischer Signaturen
erzeugt werden.
}
folgendermaßen aus:\\
Der Absender berechnet aus seiner Nachricht und seinem geheimen Schlüssel
die digitale Signatur der Nachricht. Im Vergleich zur handschriftlichen
Unterschrift hat die digitale Signatur den Vorteil, dass die
Unterschrift auch vom unterschriebenen Dokument abhängt. Die Unterschriften
ein und desselben Teilnehmers sind verschieden, sofern die unterzeichneten
Dokumente nicht vollkommen übereinstimmen. Selbst das Einfügen eines
Leerzeichens in den Text würde zu einer anderen Signatur führen. Eine
Verletzung der Nachrichtenintegrität wird also vom Empfänger der
Nachricht erkannt, da in diesem Falle die Signatur nicht mehr zum Dokument
passt und sich bei der Überprüfung als unkorrekt erweist.

Das Dokument wird samt Signatur an den Empfänger verschickt. Dieser kann
mit Hilfe des öffentlichen Schlüssels des Absenders, des Dokuments und
der Signatur feststellen, ob die Signatur korrekt ist.
Das gerade beschriebene Verfahren hätte in der Praxis jedoch einen
entscheidenden Nachteil: Die Signatur wäre ungefähr genauso lang wie das
eigentliche Dokument. Um den Datenverkehr nicht unnötig anwachsen zu
lassen und aus Performance-Gründen\index{Performance} wendet man -- vor
dem Signieren -- auf das Dokument eine kryptographische
Hashfunktion\footnote{%
Hashfunktionen\index{Hashfunktion} sind in CT1\index{CT1}
an mehreren Stellen implementiert.\\
In den Menüs \textbf{Einzelverfahren \textbackslash{} Hashverfahren} bzw.
              \textbf{Analyse \textbackslash{} Hashverfahren}
haben Sie die Möglichkeit
% hier die items nicht einrücken!%behrendt: wieso??
\begin{itemize}[nosep,label=-]
\item 6 Hashfunktionen auf den Inhalt des aktiven Fensters anzuwenden,\\
\item für eine Datei den Hashwert zu berechnen,\\
\item in der Hash-Demo die Auswirkung von Textänderungen auf den
      Hashwert zu testen,\\
\item aus einem Passwort gemäß dem PKCS\#5-Standard\index{PKCS\#5}
      einen Schlüssel zu berechnen,\\
\item aus einem Text und einem geheimen Schlüssel HMACs zu berechnen, und\\
\item aufgrund von gezielt gesuchten Hashwertkollisionen\index{Kollision}
      einen Angriff auf digitale Signaturen zu simulieren.
\end{itemize}
CT2\index{CT2} und JCT\index{JCT} enthalten ebenfalls verschiedene
Hashverfahren: Vergleiche die Funktionenliste im Anhang
\ref{s:appendix-template-overview-CT2} und
\ref{s:appendix-function-overview-JCT}.
} an. Deren Output wird dann signiert.



% --------------------------------------------------------------------------
\hypertarget{Hash-functions-ht}{}
\section{Hashfunktionen}
\index{Hashfunktion}
Eine {\em Hashfunktion}\footnote{%
Hashverfahren berechnen eine komprimierte Repräsentation
elektronischer Daten (Message).
Die Verarbeitung dieser Message durch das Hashverfahren ergibt als Output
einen sogenannten Message Digest. Message Digests sind typischerweise
zwischen 128 und 512 Bit lang -- abhängig vom Algorithmus.
Sichere Hashverfahren werden typischerweise mit anderen kryptographischen
Algorithmen kombiniert, wie z.B. Digitale-Signatur-Algorithmen,
Keyed-Hash Message Authentication Codes, oder bei der Erzeugung von
Zufallszahlen (Bits) benutzt.
}
bildet eine Nachricht beliebiger Länge auf eine Zeichenfolge mit
konstanter Größe, den \index{Hashwert}
Hashwert, ab.



% --------------------------------------------------------------------------
\subsection{Anforderungen an Hashfunktionen}

Kryptographisch sichere Hashfunktionen erfüllen folgende drei Anforderungen
(Reihenfolge so, dass die Anforderungen ansteigen):
\begin{itemize}
 \item Standhaftigkeit gegen 1st-Pre-Image-Attacks:
      \index{Pre-Image-Attack!1st} \index{Angriff!Pre-Image!1st}\\
      Es sollte praktisch unmöglich sein, zu einer gegebenen Zahl eine
      Nachricht zu finden, die genau diese Zahl als Hashwert hat.\\
      Gegeben (fix): Hashwert H',\\
      Gesucht: Nachricht m, so dass gilt: H(m) = H'.
 \item Standhaftigkeit gegen 2nd-Pre-Image-Attacks:
      \index{Pre-Image-Attack!2nd} \index{Angriff!Pre-Image!2nd}\\
      Es sollte praktisch unmöglich sein, zu einer gegebenen Nachricht
      eine zweite Nachricht zu finden, die genau denselben Hashwert hat.\\
      Gegeben (fix): Nachricht m1 [und damit der Hashwert H1 = H(m1)],\\
      Gesucht: Nachricht m2, so dass gilt: H(m2) = H1.
 \item Standhaftigkeit gegen Kollisionsangriffe:
      \index{Angriff!Kollisionsangriff}\\
      Es sollte es praktisch unmöglich sein, zwei (beliebige) Nachrichten
      mit demselben Hashwert (welcher ist egal) zu finden.\\
      Gesucht: 2 Nachrichten m1 und m2, so dass gilt: H(m1) = H(m2).
\end{itemize}




% --------------------------------------------------------------------------
\subsection{Aktuelle Angriffe gegen Hashfunktionen // SHA-3}
\label{collision-attacks-against-sha-1}

Bisher konnte die Existenz von perfekt sicheren kryptographischen
Hashfunktionen nicht formal bewiesen werden.

Über mehrere Jahre gab es keine neuen Attacken gegen Hashverfahren,
und allgemein wurde den Kandidaten, die in der Praxis bislang keine
Schwächen in ihrer Struktur gezeigt hatten
(zum Beispiel \index{SHA-1} SHA-1\footnote{%
  SHA-1 \index{SHA-1} ist eine in den Standards FIPS 180-1 (durch die
  US-Behörde NIST), ANSI X9.30 Part 2 und
  \cite{FIPS186} spezifizierte 160-Bit Hashfunktion.\\
  SHA bedeutet \glqq Secure Hash Algorithm\grqq~und wird häufig benutzt, z.B.
  mit DSA, RSA oder ECDSA.\\
  Der aktuelle Standard \cite{FIPS180} definiert vier sichere Hashverfahren
  -- SHA-1, SHA-256, SHA-384 und SHA-512.
  Für diese Hashalgorithmen sind in der Testsuite FIPS 140-2 auch
  Validierungstests definiert.

  Die Ausgabelänge der SHA-Algorithmen wurde vergrößert aufgrund der
  Möglichkeit von Geburtstagsangriffen:
  \index{Angriff!Geburtstagsangriff} \index{Kollision}
  diese machen -- grob gesprochen -- den n-Bit AES und ein 2n-bit
  Hashverfahren äquivalent:\\
  - 128-bit AES -- SHA-256\\
  - 192-bit AES -- SHA-384\\
  - 256-bit AES -- SHA-512.

  Mit CT1\index{CT1} können Sie den Geburtstagsangriff
  \index{Angriff!Geburtstagsangriff} auf digitale Signaturen
  nachvollziehen:\\
  über das Menü \textbf{Analyse \textbackslash{} Hashverfahren
  \textbackslash{} Angriff auf den Hashwert der digitalen Signatur}.\\
  CT2\index{CT2} enthält einen MD5-Kollisionsgenerator.
  }
oder \index{RIPEMD-160} RIPEMD-160\footnote{%
  RIPEMD-160, RIPEMD-128 und die optionale Erweiterung RIPEMD-256 haben
  Object Identifier, definiert von der ISO-identifizierten Organisation
  TeleTrusT, sowohl für Hashverfahren als auch in Kombination mit RSA.
  RIPEMD-160 ist Teil des internationalen ISO/IEC-Standards
  ISO/IEC 10118-3:1998 für dedizierte Hashfunktionen, zusammen mit
  RIPEMD-128 and SHA-1. Weitere Details:\\
%% BE_Link umgeleitet: - \url{http://www.esat.kuleuven.ac.be/~bosselae/ripemd160.html}\\
- \url{https://homes.esat.kuleuven.be/~bosselae/ripemd160.html}\\
- \url{http://www.ietf.org/rfc/rfc2857.txt} (``The Use of HMAC-RIPEMD-160-96
   within ESP and AH'').
  }%
) vertraut.

Auf der Crypto 2004 (August 2004)\footnote{%
    \url{http://www.iacr.org/conferences/crypto2004/} }
wurde dieses Sicherheitsgefühl jedoch stark in Zweifel gezogen:
Chinesische Wissenschaftler veröffentlichten
Kollisionsangriffe gegen MD4, SHA-0 und Teile von SHA-1, die
weltweit zu einer starken Beschäftigung mit neuen Hash-Angriffen
führte.

Die zunächst veröffentlichten Resultate reduzierten den erwarteten Aufwand für
die Suche nach einer SHA-1 Kollision von $2^{80}$ (brute-force) auf $2^{69}$
\cite{Wang2005}.  In der Folge wurden Verfahren angekündigt, die den Aufwand
weiter auf $2^{63}$ \cite{Wang2005b} und $2^{52}$ \cite{McDonald2009} reduzieren
sollen.  Damit wäre der Kollisionsangriff in den Bereich des praktisch möglichen
gerückt, denn ähnliche Aufwände wurden in der Vergangenheit schon realisiert (s.\
\ref{cm_Brute-force-versus-Symmetr}).

Die Sicherheit bereits erstellter Signaturen wird durch den geschilderten
Kollisionsangriff aber nicht gefährdet.

% be_2005_UPDATEN_if-hash-attacks-make-progress
Nach dem aktuellen Kenntnisstand ist keine Panik angesagt, aber für
digitale Signaturen sollten zumindest in Zukunft längere Hashwerte und/oder
andere Verfahren benutzt werden.

Das U.S. National Institute of Standards and Technology (NIST)\index{NIST} hat
schon vor Bekanntwerden der neuen Ergebnisse angekündigt, SHA-1 in den
näch"-sten Jahren auslaufen zu lassen. Es ist daher zu empfehlen, für neue
Produkte zur Erstellung von Sig"-naturen SHA-1 nicht mehr zu verwenden. Die
SHA-2 Familie \cite{FIPS180} bietet stärkere Verfahren.

Um den neuen Erkenntnissen in der Kryptoanalyse von Hashfunktionen Rechnung
zu tragen, hat das NIST 2008 einen Wettbewerb gestartet, um eine neue
Hash-Funktion jenseits der SHA-2-Familie zu entwickeln: Als neue Hashfunktion
\glqq SHA-3\grqq~wurde im Oktober 2012 Keccak\index{Keccak}\index{SHA-3}
verkündet.\footnote{%
\url{https://csrc.nist.gov/groups/ST/hash/sha-3/}\\
    Mit \textbf{CT2}\index{CT2} können Sie im Startcenter
    über \textbf{Vorlagen \textbackslash{} Hash-Funktionen \textbackslash{}
    Keccak-Hash (SHA-3)} die Keccak-Hashfunktion ausführen und visualisieren.\\
    Keccak kann auch als Zufallszahlengenerator und als Stromchiffre benutzt
    werden: Das finden Sie in den Startcenter-Vorlagen über
    \textbf{Werkzeuge \textbackslash{} Keccak PRNG}, bzw.
    \textbf{Kryptographie \textbackslash{} Modern \textbackslash{} Symmetrisch
         \textbackslash{} Keccak-Stromchiffre}.
}

Weitere Informationen zu diesem Thema finden sich z.B. in dem Artikel
% BE: the \ after \grqq avoids that "(2005" follows without a blank. A "~" could do it too.
\glqq Konsequenzen der erfolgreichen Angriffe auf SHA-1\grqq\
(2005) von Reinhard Wobst und Jürgen Schmidt\footnote{%
      \url{https://www.heise.de/security/artikel/Konsequenzen-der-erfolgreichen-Angriffe-auf-SHA-1-270648.html}.\\
  Weitere Quellen sind z.B.:\\
      %%% Toter Link: \url{http://www.bsi.bund.de/esig/basics/techbas/krypto/index.htm}\\
      \url{https://csrc.nist.gov/groups/ST/toolkit/index.html}.
}
  von Heise Security.




% --------------------------------------------------------------------------
\subsection{Signieren mit Hilfe von Hashfunktionen}

\begin{ctsquote}
\glqq Manipulation war Sobols Spezialität ... das Hauptziel der Ermittlungen sollte sein, hinter Sobols Masterplan zu kommen.\grqq
\caption[Daniel Suarez]{Daniel Suarez\footnotemark}\index{Suarez, Daniel}
\end{ctsquote}
\addtocounter{footnote}{0}\footnotetext{Daniel Suarez, \glqq Daemon\grqq, rororo, (c) 2010,
Kapitel 14, \glqq mem-payload\grqq, S. 148, Ross.}

Signatur-Verfahren mit Hashfunktion\footnote{%
Vergleiche auch:\\
      \url{http://de.wikipedia.org/wiki/Digitale_Signatur},\\
      \url{http://en.wikipedia.org/wiki/Digital_signature}.
}
funktionieren folgendermaßen: Anstatt das eigentliche Dokument zu signieren,
berechnet der Absender nun zuerst den Hashwert der Nachricht und signiert
diesen. Der Empfänger bildet ebenfalls den Hashwert der Nachricht (der
benutzte Algorithmus muss bekannt sein). Er überprüft dann, ob die
mitgeschickte Signatur eine korrekte Signatur des Hashwertes ist. Ist dies der
Fall, so wurde die Signatur korrekt verifiziert. Die Authentizität der
Nachricht ist damit gegeben, da wir angenommen hatten, dass aus der Kenntnis
des öffentlichen Schlüssels nicht der geheime Schlüssel abgeleitet werden
kann. Dieser geheime Schlüssel wäre jedoch notwendig, um Nachrichten in einem
fremden Namen zu signieren.

Einige digitale Signaturverfahren basieren auf asymmetrischer
Verschlüsselung, das bekannteste Beispiel dieser Gattung ist RSA. Für die
RSA-Signatur verwendet man die gleiche mathematische Operation wie zum
Entschlüsseln, nur wird sie auf den Hash-Wert des zu unterschreibenden
Dokuments angewendet.

Andere Systeme der digitalen Signatur wurden, wie DSA (Digital Signature
Algorithm), ausschließlich zu diesem Zweck entwickelt, und stehen in
keiner direkten Verbindung zu einem entsprechenden
Verschlüsselungsverfahren.

Beide Signaturverfahren, RSA und DSA, werden in den folgenden beiden
Abschnitten näher beleuchtet. Anschließend gehen wir einen Schritt weiter
und zeigen, wie basierend auf der elektronischen Unterschrift das digitale
Pendant zum Personalausweis entwickelt wurde. Dieses Verfahren nennt man
Public-Key-Zertifizierung.


% --------------------------------------------------------------------------
\section{RSA-Signatur}
\index{Signatur!digital}
\index{Signatur!RSA}
\index{RSA!Signatur}

\def\Mod#1{\ (\mbox{mod }#1)}

Wie im Kommentar am Ende von \hyperlink{RSAproof}{Abschnitt
\ref{RSAproof}} bemerkt, ist es möglich, die RSA-Operati"-onen mit dem
privaten und öffentlichen Schlüssel in umgekehrter Reihenfolge auszuführen,
d.~h.\ $M$ hoch $d$ hoch $e \Mod{N}$ ergibt wieder $M$. Wegen dieser
simplen Tatsache ist es möglich, RSA als Signaturverfahren zu
verwenden.

Eine RSA-Signatur $S$ zur die Nachricht $M$ wird durch folgende Operation
mit dem privaten Schlüssel erzeugt:
$$ S \equiv M^d \Mod{N} $$
Zur Verifikation wird die korrespondierende Public-Key-Operation auf der
Signatur $S$ ausgeführt und das Ergebnis mit der Nachricht $M$ verglichen:
$$
S^e \equiv (M^d)^e \equiv (M^e)^d \equiv M \Mod{N}$$
Wenn das Ergebnis
$S^e$ mit der Nachricht $M$ übereinstimmt, dann akzeptiert der Prüfer die
Sig"-natur, andernfalls ist die Nachricht entweder verändert worden, oder
sie wurde nicht vom Inhaber von $d$ unterschrieben.

Wie weiter oben erklärt, werden Signaturen in der Praxis nie direkt auf der
Nachricht ausführt, sondern auf einem kryptographischen Hashwert davon. Um
verschiedene Attacken auf das Sig"-naturverfahren (und seine Kombination mit
Verschlüsselung) auszuschließen, ist es nötig, den Hashwert vor der
Exponentiation auf spezielle Weise zu formatieren, wie in PKCS\#1 (Public
Key Cryptography Standard \#1 \cite{PKCS1_2002})\index{PKCS\#1} beschrieben.
Der Tatsache, dass
dieser Standard nach mehreren Jahren Einsatz revidiert werden musste, kann
als Beispiel dafür dienen, wie schwer es ist, kryptographische Details
richtig zu definieren.


% --------------------------------------------------------------------------
\section{DSA-Signatur}
\index{Signatur!digital}
\index{Signatur!DSA}
\index{DSA-Signatur}

Im August 1991 hat das U.S. National Institute of Standards and Technology
(NIST)\index{NIST} einen digitalen Signaturalgorithmus (DSA, Digital Signature
Algorithm) vorgestellt, der später zum U.S. Federal Information Processing
Standard (FIPS 186 \cite{FIPS186}) wurde.

Der Algorithmus ist eine Variante des ElGamal-Verfahrens. Seine Sicherheit
beruht auf dem Diskreten Logarithmus
Problem\index{Logarithmusproblem!diskret}. Die Bestandteile des privaten
und öffentlichen DSA-Schlüssels, sowie die Verfahren zur Signatur und
Verifikation sind in Krypto-Verfahren~\ref{dsasigproc} zusammengefasst.
\begin{cryptoprocedure}
\paragraph*{Öffentlicher Schlüssel}\strut\\
\begin{tabular}{l@{ }l}
$p$ & prim\\
$q$ & 160-Bit Primfaktor von $p - 1$\\
$g$ & $ = h^{(p-1)/q}  \mbox{ mod } p$, wobei $h < p - 1$ und
$h^{(p-1)/q} > 1  \Mod{p}$\\
$y$ & $\strut \equiv  g^x  \mbox{ mod } p$
\end{tabular}

\begin{remark}{:} Die Parameter $p,q$ und $g$ können von einer Gruppe von
Benutzern gemeinsam genutzt werden.
\end{remark}

\paragraph*{Privater Schlüssel}\strut\\
\begin{tabular}{l@{ }l}
$x < q$ (160-Bit Zahl)
\end{tabular}

\paragraph*{Signatur}\strut\\
\begin{tabular}{l@{ }l}
$m$ & zu signierende Nachricht\\
$k$ & zufällig\index{Zufall} gewählte Primzahl, kleiner als $q$\\
$r$ & $= (g^k \; \mbox{ mod } p) \mbox{ mod } q$\\
$s$ & $= (k^{-1}(\mbox{SHA-1}(m) + xr)) \mbox{ mod } q$
\end{tabular}

\begin{remark}{:}
\begin{itemize}
\item $(s,r)$ ist die Signatur.
\item Die Sicherheit der Signatur hängt nicht nur von der Mathematik ab,
sondern auch von der Verfügbarkeit einer guten Zufallsquelle\index{Zufall}
für $k$.
\item SHA-1 \index{SHA-1} ist eine 160-Bit Hashfunktion.
\end{itemize}
\end{remark}
\paragraph*{Verifikation}\strut\\
\begin{tabular}{l@{ }l}
$w$ & $= s^{-1} \;  \mbox{ mod } q$\\
$u_1$ & $= (\mbox{SHA-1}(m)w) \mbox{ mod } q$\\
$u_2$ & $= (rw)  \mbox{ mod } q$\\
$v$ & $= (g^{u_1}y^{u_2}) \mbox{ mod } p)  \mbox{ mod } q$\\

\end{tabular}

\begin{remark}{:} Wenn $v = r$, dann ist die Signatur gültig.
\end{remark}
\caption{DSA-Signatur}
\label{dsasigproc}
\end{cryptoprocedure}

Obwohl DSA unabhängig von einem Verschlüsselungsverfahren so spezifiziert
wurde, dass es aus Länder exportiert werden kann, die den Export von
kryptographischer Hard- und Software einschränken (wie die USA zum
Zeitpunkt der Spezifikation), wurde festgestellt
\cite[S.~490]{Schneier1996}, dass die Operationen des DSA dazu geeignet
sind, nach RSA bzw. ElGamal zu verschlüsseln.



% --------------------------------------------------------------------------
\section{Public-Key-Zertifizierung}
\index{Zertifizierung!Public-Key}
\index{PKI}
Ziel der Public-Key-Zertifizierung ist es, die Bindung eines
öffentlichen Schlüssels an einen Benutzers zu garantieren und nach außen
nachvollziehbar zu machen. In Fällen, in denen nicht sichergestellt werden
kann, dass ein öffentlicher Schlüssel auch wirklich zu einer bestimmten
Person gehört, sind viele Protokolle nicht mehr sicher, selbst wenn die
einzelnen kryptographischen Bausteine nicht geknackt werden können.



% --------------------------------------------------------------------------
\subsection{Die Impersonalisierungsattacke}
\label{Impersonalisierungsattacke}
\index{Impersonalisierungsattacke} \index{Angriff!Impersonalisierungsattacke}

Angenommen Charlie hat zwei Schlüsselpaare (PK1, SK1) und (PK2, SK2).
Hierbei bezeichnet SK den geheimen Schlüssel (secret key) und PK den
öffentlichen Schlüssel (public key). Weiter angenommen, es gelingt ihm,
Alice PK1 als öffentlichen Schlüssel von Bob und Bob PK2 als
öffentlichen Schlüssel von Alice \glqq unterzujubeln\grqq~(etwa indem
er ein öffentliches Schlüsselverzeich"-nis fälscht).

Dann ist folgender Angriff möglich:
\begin{itemize}
    \item Alice möchte eine Nachricht an Bob senden. Sie verschlüsselt diese
          mit PK1, da sie denkt, dies sei Bobs öffentlicher Schlüssel.
          Anschließend signiert sie die Nachricht mit ihrem geheimen
         Schlüssel und schickt sie ab.
    \item Charlie fängt die Nachricht ab, entfernt die Signatur und
          entschlüsselt die Nachricht mit SK1. Wenn er möchte, kann er die
          Nachricht anschließend nach Belieben verändern. Dann
          verschlüsselt er sie wieder, aber diesmal mit dem echten
          öffentlichen Schlüssel von Bob, den er sich aus einem
          öffentlichen Schlüsselverzeichnis geholt hat, signiert sie mit
          SK2 und schickt die Nachricht weiter an Bob.
    \item Bob überprüft die Signatur mit PK2 und wird zu dem Ergebnis
          kommen, dass die Signatur in Ordnung ist. Dann entschlüsselt er
          die Nachricht mit seinem geheimen Schlüssel.
\end{itemize}

Charlie ist so in der Lage, die Kommunikation zwischen Alice und Bob
abzuhören und die ausgetauschten Nachrichten zu verändern, ohne dass dies
von den beteiligten Personen bemerkt wird. Der Angriff funktioniert
auch, wenn Charlie nur ein Schlüsselpaar hat.

Ein anderer Name für diese Art von Angriffen ist
\index{Angriff!Man-in-the-Middle-Attack}
\glqq Man-in-the-Middle-Attack\grqq. Hilfe gegen diese Art von Angriffen
verspricht die Public-Key-Zertifizierung, die die \index{Authentizität}
Authentizität öffent"-licher Schlüssel garantieren kann. Die am weitesten
verbreitete Zertifizierungsmethode ist der X.509-Standard.

% --------------------------------------------------------------------------
\subsection{X.509-Zertifikat}
\index{X.509}
Jeder Teilnehmer, der sich per X.509-Zertifikat \cite{X.509}
die Zugehörigkeit seines öffentlichen Schlüssels zu seiner realen Person
bestätigen lassen möchte, wendet sich an eine sogenannte
\index{Certification Authority (CA)} Certification Authority (CA)\footnote{%
Oft auch Trustcenter oder im deutschen Signaturgesetz
\glqq Zertifizierungsdiensteanbieter\grqq\ genannt, wenn die Zertifikate nicht
nur einer geschlossenen Benutzergruppe angeboten werden.
}.
Dieser beweist er seine Identität (etwa durch Vorlage seines
Personalausweises). Anschließend stellt die CA ihm ein elektronisches
Dokument (Zertifikat) aus, in dem im wesentlichen der Name des
Zertifikatnehmers und der Name der CA, der öffentliche Schlüssel des
Zertifikatnehmers und der Gültigkeitszeitraum des Zertifikats vermerkt
sind. Die CA unterzeichnet das Zertifikat anschließend mit ihrem geheimen
Schlüssel.

Jeder kann nun anhand des öffentlichen Schlüssels der CA überprüfen, ob
das Zertifikat unverfälscht ist. Die CA garantiert also die Zugehörigkeit
von Benutzer und öffentlichem Schlüssel.

Dieses Verfahren ist nur so lange sicher, wie die Richtigkeit des
öffentlichen Schlüssels der CA sichergestellt ist. Aus diesem Grund lässt
jede CA ihren öffentlichen Schlüssel bei einer anderen CA zertifizieren,
die in der Hierarchie über ihr steht. In der obersten Hierarchieebene
(Wurzelinstanz) gibt es in der Regel nur eine CA, die dann natürlich keine
Möglichkeit mehr hat, sich ihren Schlüssel bei einer anderen CA
zertifizieren zu lassen. Sie ist also darauf angewiesen, ihren Schlüssel
auf andere Art und Weise gesichert zu übermitteln. Bei vielen
Software-Produkten, die mit Zertifikaten arbeiten (zum Beispiel den
Webbrowsern von Microsoft und Netscape) sind die Zertifikate dieser
Wurzel-CAs schon von Anfang an fest in das Programm eingebettet und können
auch vom Benutzer nachträglich nicht mehr geändert werden. Aber auch durch
öffentliche Bekanntgabe in Zeitungen können (öffentliche) CA-Schlüssel
gesichert übermittelt werden.



%------------------------------------------------------------------------------
\printbibliography[%
	heading=subbibintoc,
	title={Literatur zu Kapitel \thechapter},
	segment=\therefsegment,
]


Alle Links wurden am 06.08.2018 überprüft.

\end{refsegment}

% Local Variables:
% TeX-master: "../script-de.tex"
% End:
