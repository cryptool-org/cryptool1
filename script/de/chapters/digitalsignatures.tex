% ....................................................................................
%                    D I G I T A L E  S I G N A T U R E N
% ....................................................................................
\newpage
\section{Digitale Signaturen}
\index{Signaturen!digitale}
Ziel der digitalen Signatur ist es, folgende zwei Punkte zu gew"ahrleisten:
\begin{itemize}
 \item Benutzerauthentizit"at: \\
      \index{Authentizit""at!Benutzer-} Es kann "uberpr"uft werden, ob eine Nachricht tats"achlich von einer bestimmten Person stammt.
 \item Nachrichtenintegrit"at: \\
      \index{Nachrichtenintegrit""at} Es kann "uberpr"uft werden, ob die Nachricht (unterwegs) ver"andert wurde.
\end{itemize}


Zum Einsatz kommt wieder eine asymmetrische Technik (siehe Verschl"usselungsverfahren).
Ein Teilnehmer, der eine digitale Signatur f"ur ein Dokument erzeugen will, mu"s ein
Schl"usselpaar besitzen. Er benutzt seinen geheimen Schl"ussel, um Signaturen zu erzeugen,
und der Empf"anger benutzt den "offentlichen Schl"ussel des Absenders, um die Richtigkeit der Signatur
zu "uberpr"ufen. Es darf wiederum nicht m"oglich sein, aus dem "offentlichen den geheimen Schl"ussel
abzuleiten.

Im Detail sieht ein \index{Signaturverfahren} {\em Signaturverfahren} folgenderma"sen aus: \\
Der Absender berechnet aus seiner Nachricht und seinem geheimen Schl"ussel die digitale
Signatur der Nachricht. Im Vergleich zur handschriftlichen Unterschrift hat die digitale Signatur
also den Vorteil, dass die Unterschrift auch vom unterschriebenen Dokument abh"angt. Die Unterschriften
ein und desselben Teilnehmers sind verschieden, sofern die unterzeichneten Dokumente nicht vollkommen
"ubereinstimmen. Selbst das Einf"ugen eines Leerzeichens in den Text w"urde zu einer anderen Signatur
f"uhren. Eine Verletzung der Nachrichtenintegrit"at wird also vom Empf"anger der Nachricht erkannt,
da in diesem Falle die Signatur nicht mehr zum Dokument pa"st und sich bei der "Uberpr"ufung als unkorrekt
erweist.

Das Dokument wird samt Signatur an den Empf"anger verschickt. Dieser kann mit Hilfe des "offentlichen
Schl"ussels des Absenders, des Dokuments und der Signatur feststellen, ob die Signatur korrekt ist.
In der Praxis hat das gerade beschriebene Verfahren jedoch einen entscheidenden Nachteil. Die Signatur ist ungef"ahr
genauso lang wie das eigentliche Dokument. Um den Datenverkehr nicht unn"otig anwachsen zu lassen und aus 
Performance-Gr"unden benutzt man eine kryptographische Hashfunktion.

Eine kryptographische \index{Hashfunktion} {\em Hashfunktion} bildet eine Nachricht beliebiger L"ange auf eine Zeichenfolge mit
konstanter Gr"o"se (meistens 128 oder 160 Bits), den \index{Hashwert} Hashwert, ab. Es sollte praktisch unm"oglich
sein, zu einer gegebenen Zahl eine Nachricht zu finden, die genau diese Zahl als Hash\-wert hat. Ferner
sollte es praktisch unm"oglich sein, zwei Nachrichten mit dem selben Hashwert zu finden. In beiden
F"allen w"urde das Signaturverfahren Schw"achen aufweisen.

Bisher konnte die Existenz von perfekt sicheren kryptographischen Hashfunktionen nicht
formal bewiesen werden. Es gibt jedoch einige gute Kandidaten, die in der Praxis bislang keine
Schw"achen gezeigt haben (zum Beispiel \index{SHA--1} SHA-�1 oder \index{RIPEMD--160} RIPEMD�-160).

Das Verfahren mit Hashfunktion sieht folgenderma"sen aus:\\
Anstatt das eigentliche Dokument zu signieren, berechnet der Absender nun zuerst den Hashwert
der Nachricht und signiert diesen. Der Empf"anger bildet ebenfalls den Hashwert der Nachricht (der benutzte
Algorithmus mu"s bekannt sein). Er "uberpr"uft dann, ob die mitgeschickte Signatur eine korrekte Signatur des
Hashwertes ist. Ist dies der Fall, so wurde die Signatur korrekt verifiziert. Die Authentizit"at der Nachricht
ist damit gegeben, da wir angenommen hatten, dass aus der Kenntnis des "offentlichen Schl"ussels nicht der
geheime Schl"ussel abgeleitet werden kann. Dieser geheime Schl"ussel w"are jedoch notwendig, um Nachrichten
in einem fremden Namen zu signieren.

Einige digitale Signaturverfahren basieren auf asymmetrischer
Verschl"usselung, das bekannteste Beispiel dieser Gattung ist RSA. F"ur die
RSA-Signatur verwendet man die gleiche mathematische Operation wie zum
Entschl"usseln, nur wird sie auf den Hash-Wert des zu unterschreibenden
Dokuments angewendet.

Andere Systeme der digitalen Signatur wurden, wie DSA (Digital Signature
Algorithm), ausschliesslich zu diesem Zweck entwickelt, und stehen in
keiner direkten Verbindung zu einem entsprechenden
Verschl"usselungsverfahren.

Beide Signaturverfahren, RSA und DSA, werden in den folgenden beiden
Abschnitten n"aher beleuchtet. Anschliessend gehen wir einen Schritt weiter
und zeigen, wie basierend auf der elektronischen Unterschrift das digitael
Pendent zum Personalausweis entwickelt wurde. Dies Verfahren nennt man
Public Key Zertifizierung.

\subsection{RSA Signatur}
\index{Signatur!digital}
\index{RSA Signatur}

\def\Mod#1{\ (\mbox{mod }#1)}

Wie im Kommentar am Ende von \hyperlink{RSAproof}{Abschnitt
\ref{RSAproof}} bemerkt, ist es m"oglich, die RSA Operationen mit dem
privaten und "offentlichen Schl"ussel in umgekehrter Reihenfolge auszuf"uhren,
d.~h.\ $M$ hoch $d$ hoch $e \Mod{N}$ ergibt wieder $M$. Wegen dieser
simplen Tatsache ist es m"oglich, RSA als Signaturverferfahren zu
verwenden. 

Eine RSA Signatur $S$ zur die Nachricht $M$ wird durch folgende Operation
mit dem privaten Schl"ussel erzeugt:
$$ S \equiv M^d \Mod{N} $$
Zur Verifikation wird die korrespondierende Public-Key-Operation auf der
Signatur $S$ ausgef"uhrt und das Ergebnis mit der Nachricht $M$ verglichen:
$$
S^e \equiv (M^d)^e \equiv (M^e)^d \equiv M \Mod{N}$$
Wenn das Ergebnis
$S^e$ mit der Nachricht $M$ "ubereinstimmt, dann akzeptiert der Pr"ufer die
Signatur, andernfalls ist die Nachrricht entweder ver"andert worden, oder
sie wurde nie vom Inhaber von $d$ unterschrieben.

Wie weiter oben erkl"art, werden Signaturen in der Praxis nie direkt auf der
Nachricht ausf"uhrt, sondern auf einem kryptograhischen Hashwert davon. Um
verschiedene Attacken auf das Signaturverfahren (und seine Kombination mit
Verschl"usselung) auszuschliessen, ist es n"otig, den Hashwert vor der
Exponentiation auf spezielle Weise zu formatieren, wie in PKCS\#1 (Public
Key Cryptography Standard \#1 \cite{PKCS1}) beschrieben. Der Tatsache, dass
dieser Standard nach mehreren Jahren Einsatz revidiert werden musste, kann
als Beispiel daf"ur dienen, wie schwer es ist, kryptographische Details
richtig hinzubekommen.

\subsection{DSA Signatur}
\index{Signatures!digital}
\index{DSA Signatur}

Im August 1991 hat das U.S. National Institute of Standards and Technology
(NIST) einen digitalen Signaturalgorithmus (DSA, Digital Signature
Algorithm) vorgestellt, der sp"ater zum U.S. Federal Information Processing
Standard (FIPS 186 \cite{FIPS186}) wurde.

Der Algorithmus ist eine Variante des ElGamal Verfahrens. Seine Sicherheit
beruhrt auf dem Diskreten Logarithmus Problem\index{Logarithm
Problem!diskret}. Die Bestandteile des privaten und "offentlichen DSA
Schl"ussels, sowie die Verfahren zur Signatur und Verifikation, sind im
Folgenden zusammengefasst.


\paragraph{"Offentlicher Schl"ussel}\strut\\
\begin{tabular}{l@{ }l}
$p$ & prim \\
$q$ & 160bit Primfaktor von $p - 1$ \\
$g$ & $ = h^{(p-1)/q}  \mbox{ mod } p$, wobei $h < p - 1$ und
$h^{(p-1)/q} > 1  \Mod{p}$ \\
$y$ & $\strut \equiv  g^x  \mbox{ mod } p$ 
\end{tabular}

\emph{Bemerkung:} Die Parameter $p,q$ und $g$ k"onnen von einer Gruppe von
Benutzern gemeinsam genutzt werden.

\paragraph{Privater Schl"ussel}\strut\\
\begin{tabular}{l@{ }l}
$x < q$ (160bit Zahl) 
\end{tabular}

\paragraph{Signatur}\strut\\
\begin{tabular}{l@{ }l}
$m$ & zu signierende Nachricht\\
$k$ & zuf"allig gew"ahlte Primzahl, kleiner als $q$\\
$r$ & $= (g^k \mbox{ mod } p) \mbox{ mod } q$\\
$s$ & $= (k^{-1}(\mbox{SHA--1}(m) + xr)) \mbox{ mod } q$
\end{tabular}

\emph{Bemerkung:}
\begin{itemize}
\item $(s,r)$ ist die Signtur.
\item Die Sicherheit der Signatur h"angt nicht nur von der Mathematik ab,
sondern auch von der Verf"ugbarkeit einer guten Zufallsquelle f"ur $k$.
\item SHA--1 \index{SHA--1} ist eine in FIPS186 spezifizierte 160bit Hashfunktion.

\end{itemize}
\paragraph{Verifikation}\strut\\
\begin{tabular}{l@{ }l}
$w$ & $= s^{-1}  \mbox{ mod } q$\\
$u_1$ & $= (\mbox{SHA--1}(m)w) \mbox{ mod } q$\\
$u_2$ & $= (rw)  \mbox{ mod } q$\\
$v$ & $= (g^{u_1}y^{u_2}) \mbox{ mod } p)  \mbox{ mod } q$\\

\end{tabular}

\emph{Bemerkung:} Wenn $v = r$, dann ist die Signatur g"ultig.

Obwohl DSA unabh"angig von einem Verschl"usselungsverfahren so spezifiziert
wurde, dass es aus L"ander exportiert werden kann, die den Export von
kryptographischer Hard- und Software einschr"anken, wie die USA zum
Zeitpunkt der Spezifikation, wurde festgestellt
\cite[p.~490]{5Schneier1996}, dass die Operationen des DSA dazu geeignet
sind, nach RSA bzw. ElGamal zu verschl"usseln.


      



% ---------------------------------------------------------------------------------------------------------------
\subsection{Public Key--Zertifizierung}
\index{Zertifizierung!Public-Key}
Ziel der Public Key-�Zertifizierung ist es, die Bindung eines "offentlichen Schl"ussel
an einen Benutzers zu garantieren und nach au"sen nachvollziehbar zu machen. In F"allen, in denen nicht
sichergestellt werden kann, dass ein "offentlicher Schl"ussel auch wirklich zu einer bestimmten Person geh"ort,
sind viele Protokolle nicht mehr sicher, selbst wenn die einzelnen kryptographischen Bausteine nicht
geknackt werden k"onnen.




% -----------------------------------------------------------------------------------------------------------------
\subsubsection{Die Impersonalisierungsattacke}
\index{Impersonalisierungsattacke}
Angenommen Charlie hat zwei Schl"usselpaare (PK1, SK1) und (PK2, SK2). Hierbei bezeichnet SK den geheimen
Schl"ussel (secret key) und PK den "offentlichen Schl"ussel (public key). Weiter angenommen, es gelingt ihm, Alice PK1
als "offentlichen Schl"ussel von Bob und Bob PK2 als "offentlichen Schl"ussel von Alice unterzujubeln (etwa indem
er ein "offentliches Schl"usselverzeichnis f"alscht).

Dann ist folgender Angriff m"oglich:
\begin{itemize}
    \item Alice m"ochte eine Nachricht an Bob senden. Sie verschl"usselt diese mit
          PK1, da sie denkt, dies sei Bobs "offentlicher Schl"ussel. Anschlie"send signiert sie die
          Nachricht mit ihrem geheimen Schl"ussel und schickt sie ab.
    \item Charlie f"angt die Nachricht ab, entfernt die Signatur und entschl"usselt
          die Nachricht mit SK1. Wenn er m"ochte, kann er die Nachricht anschlie"send nach Belieben
          ver"andern. Dann verschl"usselt er sie wieder, aber diesmal mit dem echten "offentlichen
          Schl"ussel von Bob, den er sich aus einem "offentlichen Schl"usselverzeichnis geholt
          hat, signiert sie mit SK2 und schickt die Nachricht weiter an Bob.
    \item Bob "uberpr"uft die Signatur mit PK2 und wird zu dem Ergebnis kommen, dass die Signatur in
          Ordnung ist. Dann entschl"usselt er die Nachricht mit seinem geheimen Schl"ussel.
\end{itemize}

Charlie ist so in der Lage, die Kommunikation zwischen Alice und Bob abzuh"oren und die ausgetauschten
Nachrichten zu ver"andern, ohne dass dies von den beteiligten Personen bemerkt wird. Der Angriff funktioniert
auch, wenn Charlie nur ein Schl"usselpaar hat.

Ein anderer Name f"ur diese Art von Angriffen ist \index{Man-in-the-middle-attack} \glqq man-�in-�the-�middle-�attack\grqq. Hilfe gegen diese Art
von Angriffen verspricht die Public Key�-Zertifizierung, die die \index{Authentizit""at} Authentizit"at "offentlicher Schl"ussel
garantieren soll. Die am weitesten verbreitete Zertifizierungsmethode ist der X.509�-Standard.

% --------------------------------------------------------------------------------------------------------------------
\subsubsection{X.509}

Jeder Teilnehmer, der sich bei \index{X.509}  X.509 die Zugeh"origkeit seines "offentlichen Schl"ussels zu seiner
realen Person best"atigen lassen m"ochte, wendet sich an eine sogenannte \index{Certification Authority (CA)} Certification Authority (CA). Dieser
beweist er seine Identit"at (etwa durch Vorlage seines Personalausweises). Anschlie"send stellt die CA ihm ein
elektronisches Dokument (Zertifikat) aus, in dem im wesentlichen die Namen des Zertifikatnehmers und
der CA, der "offentliche Schl"ussel des Zertifikatnehmers und der G"ultigkeitszeitraum des Zertifikats vermerkt
sind. Die CA unterzeichnet das Zertifikat anschlie"send mit ihrem geheimen Schl"ussel.


Jeder kann nun anhand des "offentlichen Schl"ussels der CA "uberpr"ufen, ob das Zertifikat unverf"alscht ist. Die
CA garantiert also die Zugeh"origkeit von Benutzer und "offentlichem Schl"ussel.

Dieses Verfahren ist nur so lange sicher, wie die Richtigkeit des "offentlichen Schl"ussels der CA
sichergestellt ist. Aus diesem Grund l"a"st jede CA ihren "offentlichen Schl"ussel bei einer anderen CA
zertifizieren, die in der Hierarchie "uber ihr steht. In der obersten Hierarchieebene gibt es in der Regel
nur eine CA, die dann nat"urlich keine M"oglichkeit mehr hat, sich ihren Schl"ussel bei einer anderen CA
zertifizieren zu lassen. Sie ist also darauf angewiesen, ihren Schl"ussel auf andere Art und Weise gesichert
zu "ubermitteln. Bei vielen Software-�Produkten, die mit Zertifikaten arbeiten (zum Beispiel den Webbrowsern
von Microsoft und Netscape) sind die Zertifikate dieser Wurzel-�CAs schon von Anfang an fest in das Programm
eingebettet und k"onnen auch vom Benutzer nachtr"aglich nicht mehr ge"andert werden. Aber auch durch die
"offentliche Bekanntgabe k"onnen ("offentliche) CA-�Schl"ussel, insbesondere die der Wurzelinstanz,
gesichert werden.

\begin{thebibliography}{99999}
\addcontentsline{toc}{subsection}{Bibliography}
\bibitem[Schneier1996]{5Schneier1996} \index{Schneier 1996} 
    Bruce Schneier, \\
    Applied Cryptography, Protocols, Algorithms,
    and Source Code in C, Wiley, 2nd edition, 1996.

\bibitem[PKCS1]{PKCS1} RSA Laboratories\\ \index{PKCS\#1}
    PKCS \#1 v2.1 Draft 3: RSA Cryptography Standard. April 19, 2002.

\bibitem[FIPS186]{FIPS186} U.S. Departement of
    Commerce/N.I.S.T.\\\index{FIPS186}
    Entity authentication using public key cryptography. February 18 1997.
\end{thebibliography}
							  

