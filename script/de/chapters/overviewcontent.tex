% $Id: aboutcryptool.tex 3714 2016-04-08 18:34:16Z esslinger $
% !Mode:: "TeX:DE"    % Setting document mode and submode for WinEdt
% ............................................................................
%          Ü b e r b l i c k  (Text der 4. Seite, noch bevor Content)
%
% ~~~~~~~~~~~~~~~~~~~~~~~~~~~~~~~~~~~~~~~~~~~~~~~~~~~~~~~~~~~~~~~~~~~~~~~~~~~~

\clearpage
\setcounter{secnumdepth}{-1}  % Prevent this chapter title from having a number
\chapter%[Überblick]%
{Überblick über den Inhalt des CrypTool-Buchs}
\setcounter{secnumdepth}{4}  % Set back default from CT-Book-de.tex (show numbers till level 4)

\parskip 4pt
%\vskip +12 pt
Der Erfolg des Internets hat zu einer verstärkten
Forschung der damit verbundenen Technologien geführt, was auch im
Bereich Kryptographie viele neue Erkenntnisse schaffte.

In diesem {\em Buch zu den CrypTool-Programmen} \index{CrypTool} finden Sie
eher mathematisch orientierte Informationen zum Einsatz von
kryptographischen Verfahren. Zu einigen Verfahren gibt es Beispielcode,
geschrieben für das Computer-Algebra-System {\bf SageMath}\index{SageMath}
(siehe Anhang~\ref{s:appendix-using-sage}).
Die Hauptkapitel sind von verschiedenen {\bf Autoren} verfasst
(siehe Anhang~\ref{s:appendix-authors}) %\hyperlink{appendix-authors}{Autoren}
und in sich abgeschlossen. Am Ende der meisten Kapitel finden Sie
Literaturangaben und Web-Links.
Die Kapitel wurden reichlich mit {\em Fußnoten} versehen, in denen auch darauf
verwiesen wird, wie man die beschriebenen Funktionen in den verschiedenen
CrypTool-Programmen aufruft.

Das \hyperlink{Chapter_EncryptionSecDefinitions}{erste Kapitel} beschreibt
die Prinzipien der symmetrischen und asymmetrischen
\hyperlink{Chapter_EncryptionSecDefinitions}{\textbf{Verschlüsselung}} und
gibt Definitionen für deren Widerstandsfähigkeit.

Im \hyperlink{Chapter_PaperandPencil}{zweiten Kapitel} wird -- aus
didaktischen Gründen -- eine ausführliche Übersicht
über \hyperlink{Chapter_PaperandPencil}{\bf Papier- und Bleistiftverfahren}
gegeben.

Ein großer Teil des Buchs ist dem faszinierenden Thema der
\hyperlink{Chapter_Primes}{\bf Primzahlen} (Kapitel \ref{Chapter_Primes})
gewidmet.
Anhand vieler Beispiele wird in die \hyperlink{Chapter_ElementaryNT}{\bf modulare Arithmetik}
und die \hyperlink{Chapter_ElementaryNT}{\bf elementare Zahlentheorie}
(Kapitel \ref{Chapter_ElementaryNT}) eingeführt. Hier bilden die Eigenschaften
des {\bf RSA-Verfahrens} einen Schwerpunkt.

Danach erhalten Sie Einblicke in die mathematischen Konzepte und
Ideen hinter der \hyperlink{Chapter_ModernCryptography}{{\bf modernen Kryptographie}}
(Kapitel \ref{Chapter_ModernCryptography}).

%Ein \hyperlink{Chapter_Hashes-and-Digital-Signatures}{weiteres Kapitel}
Kapitel \ref{Chapter_Hashes-and-Digital-Signatures} gibt einen Überblick zum Stand der
Attacken gegen moderne \hyperlink{Chapter_Hashes-and-Digital-Signatures}{\bf Hashalgorithmen}
und widmet sich dann kurz den \hyperlink{Chapter_Hashes-and-Digital-Signatures}{\bf digitalen Signaturen}
--- sie sind unverzichtbarer Bestandteil von E-Business-Anwendungen.

Kapitel \ref{Chapter_EllipticCurves} stellt \hyperlink{Chapter_EllipticCurves}
{\bf Elliptische Kurven} vor: Sie sind eine Alternative zu RSA und für die
Implementierung auf Chipkarten besonders gut geeignet.

Kapitel \ref{Chapter_BitCiphers} führt in die \hyperlink{Chapter_BitCiphers}{\bf Boolesche Algebra} ein.
Boolesche Algebra ist Grundlage der meisten modernen, symmetrischen
Verschlüsselungsverfahren, da diese auf Bitströmen und Bitblöcken operieren.
Prinzipielle Konstruktionsmethoden dieser Verfahren werden beschrieben
und in SageMath implementiert.

Kapitel \ref{Chapter_HomomorphicCiphers} stellt
\hyperlink{Chapter_HomomorphicCiphers}{\bf Homomorphe Kryptofunktionen}
vor: Sie sind ein modernes Forschungsgebiet, das insbesondere im Zuge des
Cloud-Computing an Bedeutung gewann.

Kapitel \ref{Chapter_Dlog-FactoringDead} beschreibt
\hyperlink{Chapter_Dlog-FactoringDead}{\bf Aktuelle
Resultate zum Lösen diskreter Logarithmen und zur Faktorisierung}.
Es gibt einen breiten Überblick und Vergleich über die zur Zeit besten
Algorithmen für (a) das Berechnen diskreter Logarithmen in
verschiedenen Gruppen, für (b) das Faktorisierungsproblem und
für (c) Elliptische Kurven. Dieser Überblick wurde zusammengestellt,
nachdem ein provozierender Vortrag auf der Black Hat-Konferenz
2013 für Verunsicherung sorgte, weil er die Fortschritte bei endlichen
Körpern mit kleiner Charakteristik fälschlicherweise auf Körper
extrapolierte, die in der Realität verwendet werden.

Das \hyperlink{Chapter_Crypto2020}{letzte Kapitel}
\hyperlink{Chapter_Crypto2020}{\bf Krypto 2020}
diskutiert Bedrohungen für bestehende kryptographische Verfahren und
stellt alternative Forschungsansätze (Post-Quantum-Kryptographie)
für eine langfristige kryptographische Sicherheit vor.

Während die CrypTool-\textit{eLearning-Programme}\index{eLearning} eher den
praktischen Umgang motivieren und vermitteln, dient das \textit{Buch} dazu,
dem an Kryptographie Interessierten ein tieferes Verständnis für die
implementierten mathematischen Algorithmen zu vermitteln -- und das
didaktisch mög"-lichst gut nachvollziehbar.

Die \hyperlink{appendix-start}{\bf Anhänge}
\ref{s:appendix-menu-overview-CT1},
\ref{s:appendix-template-overview-CT2},
\ref{s:appendix-function-overview-JCT} und
\ref{s:appendix-function-overview-CTO}
erlauben einen schnellen Überblick über die Funktionen in den verschiedenen
CrypTool-Varianten\index{CT1}\index{CT2}\index{JCT}\index{CTO} via:
\begin{itemize}
  \item der Funktionsliste und
        dem \hyperlink{appendix-menu-overview-CT1}
                      {Menübaum von CrypTool~1 (CT1)},
  \item der Funktionsliste und
        den \hyperlink{appendix-template-overview-CT2}
                      {Vorlagen in CrypTool~2 (CT2)},
  \item der \hyperlink{appendix-function-overview-JCT}
                      {Funktionsliste von JCrypTool (JCT)}, und
  \item der \hyperlink{appendix-function-overview-CTO}
                      {Funktionsliste von CrypTool-Online (CTO)}.
\end{itemize}
%Anker sind \hypertarget{appendix-menutree}{} und \label{s:appendix-menutree}
%  - \hyperlink{}{} legt Link auf die Seite unter den Text in 2. Klammer
%  - \ref{} legt Link und fügt Kapitelnummer ein.

% Bernhard Esslinger, Matthias Büger, Bartol Filipovic, Henrik Koy, Roger Oyono
% und Jörg Cornelius Schneider
Die Autoren möchten sich an dieser Stelle bedanken bei den Kollegen
in der jeweiligen Firma und an den Universitäten Bochum, Darmstadt, Frankfurt,
Gießen, Karlsruhe, Lausanne, Paris und Siegen.

\enlargethispage{0.5cm}
Wie auch bei dem E-Learning-Programm CrypTool\index{CrypTool} wächst
die Qualität des Buchs mit den Anregungen und Verbesserungsvorschlägen
von Ihnen als Leser. Wir freuen uns über Ihre Rück"-mel"-dung.



% Local Variables:
% TeX-master: "../script-de.tex"
% End:
