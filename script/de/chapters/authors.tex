% $Id:
% !Mode:: "TeX:DE"    % Setting document mode and submode for WinEdt
% ..............................................................................
%             A U T O R E N
% ~~~~~~~~~~~~~~~~~~~~~~~~~~~~~~~~~~~~~~~~~~~~~~~~~~~~~~~~~~~~~~~~~~~~~~~~~~~~~~

\newpage
\hypertarget{appendix-authors}{}
\section{Autoren des CrypTool-Buchs}\index{CrypTool}
\label{s:appendix-authors}

Dieser Anhang f�hrt die Autoren\index{Autoren} dieses Dokuments auf.\\
Die Autoren sind namentlich am Anfang jedes Kapitels aufgef�hrt,
zu dem sie beigetragen haben.

\begin{description}

%% \item[Bernhard Esslinger,] \mbox{}\\  %% be_2005 Das \\ allein wird ignoriert (brauche etwas davor; "\hfill \\" ginge auch).  %% be_2016 Mit geschweiften Klammern ging es auch mit \\ allein! Der Name steht dann etwas einger�ckt.
\hypertarget{author_Bernhard-Esslinger}
\item{\bf Bernhard Esslinger} \\
Initiator des CrypTool-Projekts, Editor und Hauptautor dieses Buchs. Professor f�r IT-Sicherheit und Kryptologie an der Universit�t Siegen. Ehemals: CISO der SAP AG, und Leiter IT-Sicherheit und Leiter Crypto Competence Center bei der Deutschen Bank.\\
E-Mail: bernhard.esslinger@gmail.com, bernhard.esslinger@uni-siegen.de

---------

\hypertarget{author_Matthias-Bueger}
\item{\bf Matthias B�ger}\\
Mitautor des Kapitels~\ref{Chapter_EllipticCurves} (\glqq \nameref{Chapter_EllipticCurves}\grqq),
Research Analyst bei der Deutschen Bank.

\hypertarget{author_Bartol-Filipovic}
\item{\bf Bartol Filipovic}\\
Urspr�nglicher Autor der Elliptische-Kurven-Implementierung
in CT1 und des ent"-sprechenden Kapitels in diesem Buch.

\hypertarget{author_Martin-Franz}
\item{\bf Martin Franz}\\
Autor des Kapitels~\ref{Chapter_HomomorphicCiphers}
(\glqq \nameref{Chapter_HomomorphicCiphers}\grqq).
Forscht und arbeitet im Bereich der angewandten Kryptographie.

\hypertarget{author_Henrik-Koy}
\item{\bf Henrik Koy}\\
Hauptentwickler und Koordinator der CT1-Entwicklung
der Versionen 1.3 und 1.4, Reviewer des Buchs und \TeX{}-Guru,
Projektleiter IT und Kryptologe bei der Deutschen Bank.

\hypertarget{author_Roger-Oyono}
\item{\bf Roger Oyono}\\
Implementierer des Faktorisierungs-Dialogs in CT1 und urspr�nglicher
Autor des Kapitels~\ref{Chapter_ModernCryptography} (\glqq \nameref{Chapter_ModernCryptography}\grqq).

\hypertarget{author_Klaus-Pommerening}
\item{\bf Klaus Pommerening}\\
Autor des Kapitels~\ref{Chapter_BitCiphers} (\glqq \nameref{Chapter_BitCiphers}\grqq),
Professor f�r Mathematik und Medizinische Informatik an der Johannes-Gutenberg-Universi\-t�t. Im Ruhestand.

\hypertarget{author_Joerg-Cornelius-Schneider}
\item{\bf J�rg Cornelius Schneider}\\
Design und Long-term-Support von CrypTool, Kryptographie-Enthusiast, IT-Architekt und
Seni"-or-Projektleiter IT bei der Deutschen Bank.

\hypertarget{author_Christine-Stoetzel}
\item{\bf Christine St�tzel}\\
Diplom Wirtschaftsinformatikerin an der Universit�t Siegen.

---------

\hypertarget{author_Johannes-Buchmann}
\item{\bf Johannes Buchmann} \\
Mitautor des Kapitels~\ref{Chapter_Crypto2020} (\glqq \nameref{Chapter_Crypto2020}\grqq),
Vizepr�sident der TU Darmstadt (TUD) und Professor an den Fachbereichen
f�r Informatik und Mathematik der TUD.  Dort hat er den Lehrstuhl
f�r Theoretische Informatik (Kryptographie und Computer-Algebra) inne.


\hypertarget{author_Alexander-May}
\item{\bf Alexander May}\\
%% \item[Alexander May] \mbox{}\\
%% Wenn das \item mit eckigen Klammer (so fr�her) gebaut war, und
%% - das \hypertarget{author_Alexander-May} DAVOR stand, dann
%%   ist die Schreibweise nicht fett, sondern normal "[Alexander May,]".
%% - das \hypertarget{author_Alexander-May} DANACH stand, dann
%%   wird auf eine Stelle etwas zu weit unten verlinkt.
%% ==> Alles ok, wenn man geschweifte Klammern nach \item nutzte (be_2016)
Mitautor des Kapitels~\ref{Chapter_Crypto2020} (\glqq \nameref{Chapter_Crypto2020}\grqq)
und des Kapitel~\ref{Chapter_Dlog-FactoringDead} (``\nameref{Chapter_Dlog-FactoringDead}'').
Ordentlicher Professor am Fachbereich Mathematik (Lehrstuhl f�r Kryptologie und
IT-Sicherheit) der Ruhr-Universit�t Bochum, und zur Zeit (2014) Leiter des Horst-G�rtz Instituts
f�r IT-Sicherheit. Sein Forschungsschwerpunkt liegt bei Algorithmen f�r die Kryptoanalyse,
insbesondere auf Methoden f�r Angriffe auf das RSA-Kryptoverfahren.


\hypertarget{author_Erik-Dahmen}
\item{\bf Erik Dahmen}\\
Mitautor des Kapitels~\ref{Chapter_Crypto2020} (\glqq \nameref{Chapter_Crypto2020}\grqq),
Mitarbeiter am Lehrstuhl Theoretische Informatik (Kryptographie und
Computeralgebra) der TU Darmstadt.

\hypertarget{author_Ulrich-Vollmer}
\item{\bf Ulrich Vollmer}\\
Mitautor des Kapitels~\ref{Chapter_Crypto2020} (\glqq \nameref{Chapter_Crypto2020}\grqq),
Mitarbeiter am Lehrstuhl Theoretische Informatik (Kryptographie und
Computeralgebra) der TU Darmstadt.

---------

\hypertarget{author_Antoine-Joux}
\item{\bf Antoine Joux}\\
Mitautor des Kapitels~\ref{Chapter_Dlog-FactoringDead} (``\nameref{Chapter_Dlog-FactoringDead}'').
Antoine Joux ist der Inhaber des Cryptology Chair der Stiftung der Universit�t
Pierre et Marie Curie (Paris 6) und ein Senior Sicherheitsexperte bei CryptoExperts, Paris.
Er arbeitete in verschiedenen Gebieten der Kryptanalyse und ist die Schl�sselfigur
bei den aktuellen Fortschritten in der Berechnung diskreter Logarithmen
in K�rpern mit kleiner Charakteristik.
%In 2013, he received the prestigeous G\"odel award in Theoretical Computer Science.

\hypertarget{author_Arjen-Lenstra}
\item{\bf Arjen Lenstra}\\
Mitautor des Kapitels~\ref{Chapter_Dlog-FactoringDead} (``\nameref{Chapter_Dlog-FactoringDead}'').
Arjen Lenstra ist ordentlicher Professor an der \'Ecole Polytechnique F\'ed\'erale de
Lausanne (EPFL) und Leiter der Forschungsabteilung f�r kryptologische Algorithmen.
Er ist einer der Erfinder des derzeit besten Algorithmus f�r ganzzahlige Faktorisierung,
des Zahlk�rpersiebs (Number Field Sieve). Au�erdem war er an vielen praktischen
Faktorisierungsrekorden beteiligt.

---------

\hypertarget{author_Minh-Van-Nguyen}
\item{\bf Minh Van Nguyen}\\
SageMath-Entwickler und Reviewer des Dokuments.


\end{description}

