\documentclass[a4paper,11pt]{article}
\usepackage[german]{babel}
\usepackage{hyperref}
%\usepackage[dvips]{graphics}
%\usepackage{graphicx}
\usepackage[pdftex]{graphicx}
\usepackage{amsmath}
\usepackage{amsfonts}
\usepackage{amsthm}
\usepackage{makeidx}
%Formatierung der Fu"snoten
\usepackage[flushmargin]{footmisc}

%\usepackage{myThanks}

% \usepackage{thumbpdf}

\makeindex
\hypersetup{colorlinks=true}

%\newlength{\myFootnoteWidth}
%\newlength{\myFootnoteLabel}
%\setlength{\myFootnoteLabel}{2.2em}%  <-- can be changed to any valid value

\setlength{\textwidth}{460 pt}
\hoffset-0.7in
\newtheorem{definition}{Definition}
\newtheorem{satz}{Satz}
\newenvironment{Beweis}[1]
  {\textbf{Beweis #1} \\}
  {\hfill$\Box$ \\}

%\pdfinfo{ /Title (CrypTool Skript) /Creator (TeX) /Producer
%(pdfTeX 0.14a) /Author (Deutsch Bank AG) /CreationDate
%(D:20000920201000) }


\title{ CrypTool Skript \\ Mathematik und Kryptographie \\
Einf\"uhrung in die elementare Zahlentheorie mit Beispielen}


\author{
(c) B. Esslinger, 1998-2002 \\
Frankfurt am Main \\
besslinger@web.de \\*[8pt]
}

\begin{document}
\pagestyle{plain}
\setlength{\fboxrule}{.5mm}
\setlength{\fboxsep}{1.75mm}
\setlength{\footnotesep}{6pt}
\addtolength{\footskip}{8pt}
%\setlength{\footskip}{4cm}
%\renewcommand{\footnoterule}{\parindent0cm\rule{13cm}{.1pt}\vspace{.2cm}}
%Formatierung der Fu"snoten

%space between text and footnote 
\renewcommand\footnoterule{%
  \vspace{2em}%   <-- one line space between text and footnoterule
%\kern-3\p@
  \hrule width .4\columnwidth
 \vspace{4pt}
%kern 2.6\p@
}

%\long\def\@makefntext#1{%
%    \parindent 1em%
 %   \noindent
%    \hbox to 1.8em{\hss\@makefnmark}#1}
\maketitle

\parskip 4pt
\vskip + 30 pt
{
Dieses Dokument enth"alt nur das Kapitel "uber die elementare Zahlentheorie aus dem CrypTool-Skript.

Anhand vieler Beispiele wird in die modulare Arithmetik und die elementare
Zahlentheorie eingef"uhrt, die dann beispielhaft beim RSA-Verfahren
angewandt werden. 

W"ahrend das Programm CrypTool eher den praktischen Umgang vermittelt, dient das
Skript dazu, dem an Kryptographie Interessierten ein tieferes Verst"andnis
f"ur die implementierten mathematischen Algorithmen zu vermitteln -- und das didaktisch
m"oglichst gut nachvollziehbar.

Der {\em Autor} m"ochte sich an dieser Stelle bedanken bei den Kollegen in der
Firma und an den Universit"aten Frankfurt, Gie"sen und Karlsruhe, insbesondere
bei Dr. Peer Wichmann vom Forschungszentrum Informatik (FZI) Karlsruhe f"ur
die unkomplizierte Unterst"utzung.

% Das Skript dient dazu, dem an Kryptographie Interessierten neben der praktischen
% Anwendung von CrypTool auch ein tieferes Verst"andnis "uber die implementierten
% kryptographischen Algorithmen zu vermitteln.

\enlargethispage{0.5cm}
Wie auch bei CrypTool w"achst die Qualit"at des Skripts mit Ihren Anregungen
und Verbesserungen. Ich freue mich "uber Ihre R"uckmeldung.


% \begin{tabbing}
% Bernhard Esslinger \= \kill
% Henrik Koy         \>  ({\bf henrik.koy@db.com}) und    \\*[4pt]
% Bernhard Esslinger \> ({\bf bernhard.esslinger@db.com}).
% \end{tabbing}

\vskip +7 pt \noindent
Die aktuelle Version von CrypTool finden Sie unter \newline
  \href{http://www.CrypTool.de}{\tt http://www.CrypTool.de},~~
  \href{http://www.cryptool.com}{\tt http://www.CrypTool.com}~~ oder ~~
  \href{http://www.cryptool.org}{\tt http://www.CrypTool.org}.
%\vskip + 7 pt \noindent
%Im Readme zu CrypTool sind die Ansprechpartner f"ur dieses kostenlose Tool genannt.
}


\newpage
\tableofcontents
\addcontentsline{toc}{section}{Inhaltsverzeichnis}
\newpage
\normalsize
\parindent0cm
\parskip4pt

% .........................................................................................
%                                      E I N F � H R U N G
% ~~~~~~~~~~~~~~~~~~~~~~~~~~~~~~~~~~~~~~~~~~~~~~~~~~~~~~~~~~~~~~~~~~~~~~~~~~~~~~~~~~~~~~~~~

% --------------------------------------------------------------------------------------------------------------

\input{ElementareZahlentheorie.inc}

\addcontentsline{toc}{section}{Index} \printindex

\end{document}
